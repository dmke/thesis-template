\chapter{Ziele}
\label{chp:objectives}

\section{Anforderungen an das FIDIUS-System}
\label{chp:objectives-functions}

\authors{\HM}{\LM \and \BK \and \JF \and \DE \and \MW}

Das \f-System soll, wie es in der Motivation bereits gesagt wurde
(siehe~\ref{chp:motivation}), Angriffe auf Netze simulieren oder
ausführen, um verschiedene \acrpl{ids} zu testen und Schwachstellen
in dessen Implementierung und Konfiguration zu finden. Das System soll
es einem durchschnittlichen Administrator eines Netzes ohne Kenntnisse
über Penetration-Testing mithilfe von Methoden der \acr{ki}
ermöglichen, Angriffe gegen sein Netz auszuführen.
Mit diesen Angriffen kann der Administrator überprüfen inwieweit Angriffe
auf sein Netz von der
Sicherheitsinfrastruktur unter anderem aus \acr{ids} und
\glos{ips} erkannt und verhindert werden.
Die Ergebnisse der Angriffe auf das Netzes sollen am Ende in einem Bericht
zusammengefasst werden und sich unter anderen als pdf exportieren lassen.
 Zusätzlich soll das
\f-System, die Konfiguration von \acrpl{ids} mit einer \acr{ki} optimieren,
damit es zwar alle Angriffe erkennt, aber die False-Positive-Rate
minimiert wird.

Das System soll unter einer Open Source Lizenz für die Allgemeinheit im
Internet veröffentlicht werden, damit andere Leute sich an der Entwicklung
beteiligen können oder sie nach dem Ende des Projektes übernehmen können.

\subsection{Aufbau}
\label{chp:objectives-archi}
Es soll sich um ein verteiltes System handeln, welches aus verschiedenen Teilen
zusammengesetzt ist.
Die Core-Komponente des \f-Systems soll die \acr{ki} enthalten, das Wissen
speichern und die Aktionen ausführen.
Die \acr{gui} zum Bedienen des Systems soll sich über eine Netzschnittstelle mit dem 
Core verbinden und sich über einem Webbrowser bedienen lassen.
Auf dieser Strecke soll \acr{xml}-\acr{rpc} verwendet werden, damit die Schnittstelle
sprachunabhängig ist.
Die Systeme sollen alle modular aufgebaut werden, damit sich einzelne
Komponenten leicht austauschen lassen und es sollen einheitliche Schnittstellen
verwendet werden, damit neue Funktionen einfach, auch ohne das Ändern von
bestehenden Code, hinzugefügt werden können.
Einige Funktionen sollen als Bibliothek implementiert werden, damit sie später
auch von anderen Programmen in einem anderen Zusammenhang benutzt werden können.
Zum Ausführen von Exploits soll \glos{msf} benutzt werden, es soll aber später
möglich sein andere ähnlich Systeme zum Ausführen von Exploits zu verwenden.
Zwischen dem Core und \glos{msf} soll sich eine weiter Netzschnittstelle
befinden, damit \glos{msf} nicht neu geladen werden muss, wenn der Core neu
geladen wird, da das Laden von \glos{msf} auch auf schnellen Computer über 20 Sekunden dauert.

%Das System besteht aus dem Core (siehe~\ref{sec:core}) welcher mithilfe von \glos{msf}
%die Exploits ausführt.
%Der Benutzer steuert den Core über die GUI des C\&C-Server (siehe~\ref{sec:candc}).
%Die beiden Teile kommunizieren über eine XMLRPC Schnittstelle miteinander und können
%auf unterschiedlichen Computer betrieben werden.

%Der Core ist modular aufgebaut und lässt sich erweitern.
%Die Decisionkomponente des Cores schlägt mithilfe einer \acr{ki} die nächste Aktionen
%vor, die das System ausführen soll.


\subsection{Hosts Angreifen}
\authors{\HM}{\BK \and \JF \and \DE \and \LM \and \MW}
\label{chp:objectives-functions-pwnen}

Das Angreifen eines Netzes besteht aus mehreren Schritten. 
Zuerst soll der Benutzer angeben, welches Netz er scannen möchte. Die gefundenen
Hosts mit ihren offenen Ports sollen dann dem Wissen des Cores hinzugefügt werden.
Dann soll die \acr{ki} das nächste anzugreifende System, anhand 
des erlangten Wissens, auswählen.

Als nächstes entscheidet die \acr{ki} auf Basis der offenen Ports und der
Versionen der dort laufenden Dienste, welcher Exploit für den Angriff benutzt werden soll.
Das \f-System soll Exploits verwenden, die offene Ports von Systemen
angreifen, da diese Angriffe am besten von \glospl{ids}n erkannt werden.
Nachdem der Angriff ausgeführt wurde sollen die erlangten Informationen, wie
welcher Exploit erfolgreich war und sonstige Informationen die über die
Reconnaissance (siehe Abschnitt~\ref{chp:objectives-anwendungsfaelle}) erlangt wurden,
in die Wissensbasis des Systems eingefügt werden.
Das System muss so konfiguriert werden können, dass Schäden an nicht
testbezogenen Systemen verhindert werden. Hierzu sollte es eine Benutzerabfrage
vor jedem Angriff geben. Außerdem sollte es möglich sein diese Abfrage in
Testnetzen, ohne Zugang zu Produktivsystemen, zu deaktivieren.

Zum Ausführen der Angriffe soll ein externes Framework
verwendet werden, welches viele Exploits enthält, 
die sich über eine einheitliche Schnittstelle ansprechen lassen.
Diese Eigenschaften werden vom \acr{msf} erfüllt, welches verwendet werden soll.
Mithilfe von \glos{msf} soll es möglich werden, \glos{pivoting} auszuführen,
also über einen Host, welcher unter der Kontrolle von \f steht einen
weiteren Host anzugreifen zu dem keine direkte Netzverbindung besteht.
Es soll auch möglich sein einen Host über einen Exploit in einer Client
Anwendung wie einem Browser anzugreifen, indem \zB der Nutzer des anzugreifenden
Hosts auf eine manipulierte Website gelockt wird.

\subsection{KI}
\authors{\DK}{\LM \and \DE \and \MW \and \JF}
\label{chp:objectives-functions-ki}

Die Anforderungen an die Künstliche Intelligenz können in zwei
Abschnitte geteilt werden: Black-Box- und Gray-Box-Szenario.

Im Black-Box-Szenario ist zu Beginn des Angriffs nichts über das Netz
bekannt. Die Künstliche Intelligenz soll entweder den Angriff
möglichst autonom durchführen können oder den Nutzer unterstützen.
Dazu muss die \acr{ki} in der Lage sein, Aktionen zu wählen und deren
Erfolg zu messen. Erst im Verlauf eines Angriffs wird neues Wissen
über das Netz erschlossen, also muss die \acr{ki} dazulernen
können. 

Im Gray-Box-Szenario ist die Topologie des Netzes bekannt, 
anderes Wissen, wie laufende Dienste oder installierte
Webserver, nicht. Diese \acr{ki}-Komponente soll in der Lage sein,
Angriffspläne zu erstellen, wobei nicht das gesamte Wissen vorliegt.
An bestimmten Stellen im Plan sollen daher sogenannte \enquote{sensing
actions} eingefügt werden, die es dem System ermöglichen, in
bestimmten Situationen neues Wissen durch Messungen zu bekommen.
Wie die Entscheidung des nächsten anzugreifenden Hosts getroffen wird
und wie die Plankomponente implementierbar ist, wird in Abschnitt~\ref{sec:ki}
beschrieben. 

\subsection{IDS testen}
\label{chp:objectives-functions-idstest}
\authors{\HM}{\BK \and \JF \and \DE \and \LM \and \MW}

Zum Testen von \glos{ids} soll das \f-System  Angriffe auf Hosts in Netzen
ausführen. Ziel soll es sein, die von einem \glos{ids} erstellten Daten, zu
analysieren die bei einem Angriff erzeugt werden. 
Das \f-System soll einerseits zum Testen der Implementierung eines \glos{ids}
verwendet werden können \zB um sie mit anderen Systemen zu vergleichen oder
zur Unterstützung bei der Entwicklung eines neuen \glos{ids}.

Die Konfiguration eines \glos{ids} ist sehr komplex und es verlangt sehr viel
Erfahrung die Konfiguration an ein bestimmtes Netz anzupassen.
Hier soll das \f-System dazu verwendet werden können, solche Konfigurationen
zu testen und sie mittels \acr{ki}-Algorithmen zu optimieren.

Bei den Tests soll versucht werden die Erkennungsrate zu erhöhen, sodass 
alle Angriffe von dem \glos{ids} erkannt werden, die \glos{false-positive}-Rate
soll aber gleichzeitig minimiert werden.

Damit das \f-System überprüfen kann, ob die Angriffe von dem zu testenden
\glos{ids} erkannt wurden, soll es möglich sein das \glos{ids} mit dem \f-System zu
verbinden, damit das \f-System die vom \glos{ids} generierten Events analysieren kann.
Über eine Schnittstelle zwischen dem \f-System und dem \glos{ids} soll
auch überprüft werden, ob keine Alarme von dem \glos{ids} generiert werden,
wenn keine Angriffe durchgeführt werden.
Zum \glos{prelude-ids} und zum \glos{ids} des \acr{fides}-Projektes soll eine
Anbindung an das \f-System erstellt werden, über welche das \f-System auf die
vom \glos{ids} generierten Events zur Analyse zugegriffen werden kann.
Schließlich soll das System des \acr{fides}-Projektes mit anderen etablierten \glos{ids}
wie dem \glos{prelude-ids} hinsichtlich der Erkennungs- und 
\glos{false-positive}rate mithilfe des \f-Systems verglichen werden.

\subsection{Reconnaissance}
\label{chp:objectives-functions-reconnaissance}
\authors{\HM}{\BK \and \JF \and \DE \and \LM}

Ziel der Reconnaissances ist es, Informationen über einen Host und seine Umgebung
zu sammeln, nachdem das \f-System die Kontrolle über ihn übernommen hat.
Hierbei ist zwischen Host-Reconnaissance und Netz-Reconnaissance zu unterscheiden.
Bei der Host-Reconnaissance soll die genaue Betriebssystemversion, seine
Netzkonfiguration und die auf dem System gespeicherten Passwörter ermittelt
werden und in Wissensbasis des \f-Systems übernommen werden.
In der Netz-Reconnaissance sollen die Netze in der Umgebung des übernommenen
Hosts über diesen gescannt werden, falls dieser Host
Zugriff zu Netzen hat, die vom \f-System nicht direkt erreichbar sind.
Diese Scans sollen sowohl passiv, durch das Mitschneiden der ankommenden Pakete auf
den Netzinterfaces, als auch über einen aktiven Scan, wie einen
Ping-Scan durchgeführt werden können.

\section{Anwendungsfälle}
\label{chp:objectives-anwendungsfaelle}
\authors{\HM}{\BK \and \JF \and \DE \and \LM}

Das \f-System ist hauptsächlich für zwei Anwendungsfälle ausgelegt.

\begin{enumerate}
  \item Bei der Entwicklung eines neuen \glos{ids} oder bei der Weiterentwicklung
    eines bestehenden \glos{ids}. Hierbei kann das \f-System in einem Testnetz
    die Funktionalität des \glos{ids} testen und es kann eine echte Umgebung
    simuliert werden.  Das \f-System kann in einem \acr{ci}-System benutzt
    werden, mit den kontinuierlich überprüft wird, ob das \glos{ids} noch
    alle Angriffe erkennt, die es vorher erkannt hat und ob die
    False-Positive-Rate sich verändert hat.
  \item Bei der Installation und Konfiguration eines \glos{ids} in einem Netz.
    Wenn ein neues \glos{ids} zum Schutz eines Netzes (\zB in einem Unternehmen)
    eingerichtet wird, kann mit dem \f-System die Erkennungsrate und der Anteil
    der False-Positive-Meldungen überprüft werden. Das \f-System kann auch
    direkt zur Optimierung der Konfiguration benutzt werden, nachdem alle
    Sensoren in einem Netz installiert wurden.
\end{enumerate}

