\makeatletter

% basically: "\maybespaceafter{foo}{bar}" defines "\foo" which prints "bar"
\def\maybespaceafter#1#2{%
  \expandafter\def\csname #1\endcsname{#2\xspace}%
}

%% common macros
\maybespaceafter{eg}{e.\,g.}

\def\diplomaname{Diploma Thesis}
\def\examinername{Examiner}
\def\supervisorname{Supervisor}


\def\diplomaname{Diplomarbeit}
\def\abstractname{Zusammenfassung}
\def\examinername{1. Gutachter}
\def\supervisorname{2. Gutachter}
\def\advisorname{Betreuer}
\def\matriculationname{Matrikel-Nr.}
\def\contentsname{Inhalt}
\def\abbreviationsname{Liste der Abkürzungen}
\def\glossaryname{Glossar}

\maybespaceafter{zB}{z.\,B.}%
\maybespaceafter{dh}{d.\,h.}%
\maybespaceafter{uU}{u.\,U.}%
\maybespaceafter{uA}{u.\,A.}%
\maybespaceafter{su}{s.\,u.}%
\maybespaceafter{idR}{i.\,d.\,R.}%

\def\monthyear{%
  \ifcase\month\or Januar\or Februar\or März\or%
    April\or Mai\or Juni\or Juli\or August\or%
    September\or Oktober\or November\or%
    Dezember\fi~\the\year\xspace%
}


\newcommand*{\acr}[1]{\gls{acr:#1}}
\newcommand*{\Acr}[1]{\Gls{acr:#1}}
\newcommand*{\ACR}[1]{\GLS{acr:#1}}
\newcommand*{\acrpl}[1]{\glspl{acr:#1}}
\newcommand*{\Acrpl}[1]{\Glspl{acr:#1}}
\newcommand*{\ACRpl}[1]{\GLSpl{acr:#1}}

\newcommand*{\glos}[1]{\gls{glos:#1}}
\newcommand*{\Glos}[1]{\Gls{glos:#1}}
\newcommand*{\GLOS}[1]{\GLS{glos:#1}}
\newcommand*{\glospl}[1]{\glspl{glos:#1}}
\newcommand*{\Glospl}[1]{\Glspl{glos:#1}}
\newcommand*{\GLOSpl}[1]{\GLSpl{glos:#1}}

%% math tools

\def\N{\ensuremath{\mathbb{N}}}
\def\Z{\ensuremath{\mathbb{Z}}}
\def\I{\ensuremath{\mathbb{I}}}
\def\Q{\ensuremath{\mathbb{Q}}}
\def\R{\ensuremath{\mathbb{R}}}
\def\C{\ensuremath{\mathbb{C}}}
\DeclareMathOperator{\sgn}{\text{sgn}}

% acronyms style
\newglossarystyle{thesisacr}{%
  \renewenvironment{theglossary}%
    {\begin{description}[style=sameline]}
    {\end{description}}
  \renewcommand*{\glossaryheader}{}%
  \renewcommand*{\glsgroupheading}[1]{}%
  \renewcommand*{\glossaryentryfield}[5]{%
      \item[\glstarget{##1}{##2}] ##3, S.~##5
  }%
  \renewcommand*{\glsgroupskip}{\indexspace}%
}

% glossary style
\newglossarystyle{thesis}{%
  \renewenvironment{theglossary}%
    {\begin{description}}
    {\end{description}}
  \renewcommand*{\glossaryheader}{}%
  \renewcommand*{\glsgroupheading}[1]{}%
  \renewcommand*{\glossaryentryfield}[5]{%
      \item[\glstarget{##1}{##2}] ##3 \\
      S.~##5
  }%
  \renewcommand*{\glsgroupskip}{\indexspace}%
}

%% some tolerance parameters
\tolerance=400
\hfuzz=0.2pt

%% do not change these lines
\makeatother
\endinput
%% anything below this line is ignored
