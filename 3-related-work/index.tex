\chapter{Related Work}
\label{chp:related-work}

\section{Angriffstools}
\authors{\DE}{\LM \and \MW}

Es existieren diverse Tools, die einen Benutzer beim Testen der
eigenen Sicherheitsmaßnahmen unterstützten. 
Man muss unterscheiden zwischen Werkzeugen, die einzelne Schritte zum
Angreifen eines Netzes erleichtern und Frameworks oder Programme, welche diese 
Tools in einen sinnvollen Zusammenhang bringen, so dass ein Benutzer sich 
nicht mehr mit einem bestimmten Tool auskennen muss, sondern von
diesen Tools so abstrahiert wird, dass ein Benutzer sich durch eine
\acr{gui} klicken kann.

Wir benutzen die freie Version des Metasploit
Frameworks\footnote{\url{http://www.metasploit.com}} als Basis. Es
gibt bereits diverse Frameworks, die einen ähnlichen Aufbau haben, wie
das \f-System. 

Hier ist die professionelle Version von
Metasploit\footnote{\url{http://www.rapid7.com/products/metasploit-pro.jsp}}
ebenso zu erwähnen, wie
Matra\footnote{\url{http://www.getmantra.com/}} und
Armitage\footnote{\url{http://www.fastandeasyhacking.com/}}. Alle
erlauben das automatisierte Testen von Sicherheitslücken in Diensten
von Hosts innerhalb von Netzen.


\section{Künstliche Intelligenz}
\authors{\DK}{\DE \and \LM \and \MW}

Neben einer Plan-Domäne aus dem FIDeS-Projekt\footnote{http://www.fides-security.org/}
von Mirko Horstmann und einer von Stefan Edelkamp existieren noch
Sicherheitsdomänen von S.\,K.~Ghosh, und Shackleton.  Diese Domänen scheinen
alle erweiterungswürdig bzw. die Domäne von Shackalton berücksichtigt auch
\enquote{social engineering}.  Da wir ein automatisiertes, unterstützendes
System bauen wollen, ist \enquote{social engineering} in unserem Kontext
nicht möglich, da das Ergebnis nicht deterministisch ist.  Eine der ersten
Domänen die für Angriffe auf Rechnernetze geschrieben wurde stammt von
Boddy \cite{BODY}.  Auch diese Domäne enthält \enquote{social engineering}
und ist für klassische Planer geschrieben.  Des Weiteren ist die
Grundannahme in der Arbeit von Boddy ein allwissender Angreifer. Da
unser Fokus auf \enquote{Gray}- beziehungsweise \enquote{Black-Box}-Tests liegt,
ist die Annahme in unserem Fall nicht zutreffend.  In der Arbeit von
Luceangeli~\cite{lucean}, beschreiben die Autoren ein weiteres
Planszenario. In deren Arbeit werden Informationen aus einem
\enquote{penetration testing tool} in eine Plandomäne übersetzt. Auch diese
Arbeit geht von einem allwissenden Angreifer aus.
Da wir \enquote{contingnet planning} nutzen, ist unser Ergebnis ähnlich zu
den Ergebnissen aus der Arbeit von Florian Junge \cite{JUNGE}.
