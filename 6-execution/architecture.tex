\section{FIDIUS-Architektur}
\label{sec:architecture}
\authors{\LM \and \DE}{}

Das \f-System basiert auf verschiedenen Komponeten die miteinander agieren.
Das zentrale Element ist der \f-Core, dieser besteht aus den Komponenten
Knowledge, Decision und Action, eine genaue Erklärung folgt in
Abschnitt~\ref{sec:core}.

Über \glos{xmlrpc} kann der \f-Core von einer \acr{gui} angesprochen werden
und Befehle und Anfragen entgegennehmen. Wir verwenden zur Bereitstellung
der \acr{gui} den C\&C-Server, dies ist eine \glos{rails}-Webanwendung die
im Abschnitt~\ref{sec:candc} genauer erläutert wird.

Als Angriffskomponente wird auf \acr{msf} zurückgegriffen. Vom \f-Core
kann mit \acr{drb} auf den MSF-DRb-Daemon zugegriffen werden, der \acr{msf}
bereitstellt. Sowohl \acr{msf} als auch der MSF-DRb-Daemon werden im
Abschnitt~\ref{sec:msf_framework} vorgestellt.   

Die EvasionDB dient dazu, \acr{ids}-Events mit durchgeführten Angriffen zu
korrelieren. Dazu wird die EvasionDB sowohl in der Angriffskomponente, bei
uns als \acr{msf}, als auch in den überwachenden \acr{siems}, wie dem
\glos{prelude-ids}, angebunden. Eine Erläuterung der EvasionDB folgt im
Abschnitt~\ref{sec:evasion-db}.

Die Komponente Snortor wurde zu zwei Zwecken entwickelt. Zum einen
dient sie dazu alle Exploits, die dem \f-System zur Verfügung stehen
nacheinander an einen Host zu senden. Zum anderen kann sie eine
\gls{glos:snort}-Instanz konfigurieren. So kann der Snortor alle Exploits
senden und über die EvasionDB abfragen, welche Events das Exploit
zur Folge hatte. Die genaue Funktionsweise ist beschrieben im
Abschnitt~\ref{snortor}.

Die Architektur des \f-System ist in Abbildung~\ref{fig:core} schematisch dargestellt:

\begin{figure}
  \centering
  \includegraphics[width=.6\linewidth]{images/core-architecture}
  \caption{Schema des \f-Systems}
  \label{fig:core}
\end{figure}
