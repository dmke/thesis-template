\section{VM-Steuerung}
\label{sec:test_vm_control}
\authors{\DH \and \LM}{\MW \and \JF}

Die \acr{vm}-Steuerung wird mit Hilfe vom \acr{msf}
und dem Lab-Plugin, welches im Abschnitt~\ref{lab} beschrieben ist,
realisiert. Für das \f-System werden zum Testen mehrere
Rechnernetze und einige Rechner benötigt. Da diese Infrastruktur nicht
mit realer Hardware umgesetzt werden konnte, weil das Equipment doch
sehr teuer ist, wurde auf Virtualisierung zurückgegriffen. Ein Vorteil
der virtualisierten Lösung ist die softwareseitige Steuerung des
Start- und Stopp-Mechanismus von \acrpl{vm}. Auch werden
Sicherungspunkte (\glospl{snapshot}) ermöglicht, die ein laufendes System
einfrieren und diesen Zustand später wiederherstellen können.

Durch die Verwendung und Erweiterung von Lab ist es uns möglich, die \acrpl{vm}
zu steuern. Allerdings muss die \acr{vm}-Steuerung auch bei den Tests
entsprechend umgesetzt sein. Es müssen zuerst einige vorbereitende Schritte
getroffen werden, um \acrpl{vm} erfolgreich für die Tests verwenden zu können. 
Zum Ausführen von \glospl{exploit} auf einer \acr{vm} wird \zB die 
\acr{ip}-Adresse für die anzugreifende \acr{vm} benötigt. Da die Entwickler 
nicht immer gezwungen werden sollen auf die Testlandschaft zum Testen des 
\f-System zurückzugreifen, müssen die \acrpl{vm} zum Ausführen der Tests 
konfiguriert werden.

Zur Konfiguration wird ein Identifikator für den Konfigurationseintrag,
den Lab-Name mit den Lab die \acr{vm} identifizieren und ansprechen kann, die
\acr{ip}-Adresse und eine Wartezeit zum starten der \acr{vm} festgelegt.
Ein typischer Konfigurationseintrag sieht dann wie folgt aus:

\begin{lstlisting}[language=Ruby]
  :xp_vm: 
    :lab_name: WINXP_SP2
    :ip: 192.168.56.101
    :waittime: 10
\end{lstlisting}

\begin{itemize}
  \item \texttt{:xp\_vm:} \acr{vm}-Identifikator der innerhalb der Tests
      genutzt wird, um den zu verwendenden Konfigurationseintrag zu bestimmen.
  \item \texttt{:lab\_name:} Der Identifikator der \acr{vm}, mit dem Lab
      die \acr{vm} ansteuern kann. Daher muss dieser Eintrag identisch zum in
      Lab verwendeten Identifikator sein.
  \item \texttt{:ip:} Die \acr{ip}-Adresse der \acr{vm} dient zum
      Identifizieren des anzugreifenden Host-Assets.
  \item \texttt{:waittime:} Eine Wartezeit, die bestimmt, wie lange das
      \f-System zum Starten und Stoppen der \acr{vm} pausieren soll.
\end{itemize}

Um die Verwendung der \acr{vm}-Steuerung noch weiter zu vereinfachen, wurden
Hilfsfunktionen implementiert, bei denen nur der \acr{vm}-Identifikator 
angegeben wird und die weiteren Einstellungen automatisch aus der Konfiguration 
gesucht werden.

Da nicht jeder Entwickler alle benötigten \acrpl{vm} zur Testdurchführung
lokal zur Verfügung hat, wird ein Test nur ausgeführt, wenn alle benötigten
\acrpl{vm} in der Konfigurationsdatei konfiguriert wurden. Dieses Vorgehen
erleichtert das Testen, da nur die Tests ausgeführt werden, deren
benötigte Testumgebung bereitgestellt werden kann. Dadurch wird die
Ausführung von Tests verhindert, die aufgrund der fehlenden Testumgebung
nur fehlschlagen können und somit eine Zeitersparnis erreicht. Außerdem
erleichtert es das Auswerten des Tests, da bei einem gemeldeten Fehlschlag
nicht erst kontrolliert werden muss, ob der Test aufgrund der Testumgebung
oder einem Fehler nicht funktioniert.

