\section[Netflow \& IPFIX -- Grundlagen]
  {Netflow \& IPFIX\\Grundlagen}
\label{sec:compositions:netflow-basics}
\authors{\AB}{}

\subsection{Einleitung}

Im Rahmen der neuen Zielfindung des Projektes \f kam die Idee auf, im
Bereich Intrusion Detection einen neuen Weg einzuschlagen. Während in den ersten
beiden Semestern auf den von Snort erzeugten Events gearbeitet werden sollte,
wurde für die zweite Projekt-Hälfte der Ansatz vorgeschlagen Netflow Daten
auszuwerten und anhand dieser Anomalien in einem Rechnernetz zu erkennen.

Dieses Referat soll erklären was sich hinter Netflow und IPFIX verbirgt, welche
Daten diese Protokolle liefern können und wie die Informationsgewinnung
funktioniert. Als Beispiel dafür dient die von Ethereal entwickelte 
Netzverkehr-Analyse-Software \texttt{nTop}, die unter anderem Netflow Daten sammeln und
in verschiedenen Sichtweisen einem Benutzer zur Verfügung stellen kann.

In einem weiteren Referat -- \enquote{Netflow / IPFIX Security und Diagnose} -- wird näher
darauf eingegangen, wie Netflow Daten zur Erkennung von Anomalien verwendet
werden können.

\subsection{Überblick}
\label{sec:compositions:netflow-basics:overview}

Bei Netflow handelt es sich um ein von Cisco entwickeltes Protokoll, das in
seiner ersten Version 1996 erschienen ist (vgl.~\cite[2]{netflow-oslebo}).
Es dient der Überwachung des IP-Verkehrs in Netzen, in denen Sensoren - sog.
Probes - installiert werden. Diese senden Informationen über den Verkehr an
Daten-Sammelstellen - sogenannte Collectoren, welche die Daten speichern,
weiterverarbeiten oder den Benutzern zur Verfügung stellen.

Insgesamt wurden 10 Netflow Versionen entwickelt, von denen jedoch nur 7
veröffentlicht wurden. Tabelle~\ref{tab:compositions:netflow-versions} zeigt
eine Übersicht der Eigenschaften. (vgl.~\cite{netflow-caligare})

\begin{longtable}{lp{.7\linewidth}}
  \rowcolor{Beige}
    Version & Eigenschaften \\
  \endhead
    \caption[]{Netflow Versionen\\\tabelletbcname}
  \endfoot
    \caption{Netflow Versionen\label{tab:compositions:netflow-versions}}
  \endlastfoot  
  V1 & Beschränkung auf IPv4\\
  V5 & IPv4, Border Gateway Protocol (BGP) und Autonomous
System (AS) Informationen, Sequenz-Nummern.\\
  V6 & Entwicklung für speziellen Kunden, wird nicht verwendet \\
  V7 & Verwendung ausschließlich auf Cisco Catalyst 5000 Switches\\
  V8 & Reduzierung des Export Volumens durch Router-Based Netflow Aggregation 
\cite{netflow-cisco-flow-aggregation}\\
  V9 & Einführung von Templates, IPv6, Multicasts, IPSec. MPLS \\
  V10 (IPFIX) & Erweiterung von V9, IETF Standardisiert, standard-export über
SCTP (statt UDP)
\end{longtable} 

Die durch Netflow-Probes gesammelten Daten sind vielseitig verwendbar. Anhand
der Informationen über den stattgefundenen Netz-Verkehr wäre beispielsweise die
Kosten Abrechnung eines Providers möglich, der anhand des aufgekommenen
Datenvolumens Rechnungen an seine Klienten ausstellt. Es lassen sich ebenso
Netztopologien erstellen anhand derer ersichtlich wird, wo im Netz wichtige
Knotenpunkte vorhanden sind die besonders belastet werden oder selber viel
Verkehr erzeugen.

Im Kontext von \f sind hierbei Sicherheitsaspekte
(siehe~\cite[6]{netflow-nvisionip-security-visualizations}) besonders
interessant, denkbar wäre unter anderem die Erkennung von:

\begin{itemize}
  \item Wrum-infizierten Rechnern, die atypischen Verkehr erzeugen.
  \begin{itemize}
    \item z.B. ein vom Slammer Wurm infizierter Mail-Server, der ständig
UDP-Pakete an Port 1434 schickt.
  \end{itemize}
  \item Übernommene Systeme, die Teil eines Botnetzes sind.
  \begin{itemize}
    \item Erkennung von erhöhtem Traffic auf Port 6667 \& 6668
  \end{itemize}
  \item Port Scans, die möglicherweise Vorbereitung eines Angriffs sind.
  \begin{itemize}
    \item Flows die eine vielzahl an Ports eines oder mehrerer Zielsysteme
erreichen
  \end{itemize}
  \item (Distributed) Denial of Service Angriffe.
  \begin{itemize}
    \item Ausgehende Angriffe: gleichzeitiges Ansteigen des Traffics auf vielen
Hosts. Eingehende Angriffe: Hohes eingehendes Datenvolumen bei dem angegriffenen
Host.
  \end{itemize}
\end{itemize}

\subsection{Funktionsweise}

Im Folgenden wird die Funktionsweise von NetFlow erläutert. Dazu gehört die Art,
in der Informationen gesammelt werden können und die Infrastruktur die dazu
benötigt wird.

\subsubsection{Flows}

Das Grundprinzip, das NetFlow zugrunde liegt, ist die Zusammenlegung des
IP-Vekehrs in sogenannte Flows. Die Einteilung wird anhand verschiedener
Parameter vorgenommen, die als Flow-Keys bezeichnet werden. Während die
Flow-Keys in Netflow V1 bis V8 eine fixe Menge bilden, aus der eine Teilnemge
ausgewählt werden kann, bieten V9 und IPFIX die Verwendung von Templates, mit
deren Hilfe eigene Flow-Keys definiert werden können.

Die Einteilung eines Flows geschieht in NetFlow V5 üblicherweise anhand eines
7-Tupels (vgl. \cite{netflow-cisco-flow-information}) bestehend aus:

\begin{itemize}
  \item IP-Adresse (Quelle \& Ziel)
  \item Port-Nummer (Quelle \& Ziel)
  \item IP Protokoll
  \item Type of Service
  \item Input logical Interface
\end{itemize}

Eine NetFlow-Probe fasst Pakete die in ihren Flow-Keys übereinstimmen zu einem
Flow zusammen und sendet die gesammelten Informationen an einen oder mehrere
Collectoren. Dabei gibt es unterschiedliche Ursachen wann ein Flow beendet wird.
Bei verbindungsorientierten Protokollen kann dies die Beendigung der Verbindung
sein, was in der Probe bei TCP anhand des FIN-Pakets erkannt werden kann.
Weiterhin kann ein Flow beendet werden, wenn nach einem definierten Timeout
keine neuen Pakete eingetroffen sind, die diesem Flow zugeordnet werden können.
Bei lang anhaltenden Flows können diese nach einem Zeitintervall beendet werden.
Anschließend entsteht ein neuer Flow mit denselben Flow-Keys. Dadurch werden
lange Flows in feste Intervalle aufgeteilt damit die gesammelten Informationen
bereits an den Collector übertragen werden können, obwohl der Flow eigentlich
noch anhält. Sowohl der Timeout für Flows ohne neue Pakete als auch der
Intervall bei lang anhaltenden Flows sind variabel und in der NetFlow-Probe
konfigurierbar.

Neben der normalen Erfassung aller Pakete besteht die Möglichkeit des Paket
Samplings. Dabei werden nurnoch Stichproben der Pakete genommen, wobei die Art
und Granularität vom Benutzer bestimmt werden.
Tabelle~\ref{tab:compositions:netflow-sampling} zeigt 3 mögliche Varianten des
Samplings, wobei \textbf{N} vom Benutzer gewählt werden kann. (siehe
\cite{netflow-cisco-flow-sampling})

\begin{longtable}{lp{.7\linewidth}}
  \rowcolor{Beige}
    Methode & Beschreibung \\
  \endhead
    \caption[]{Paket-Sampling Varianten\\\tabelletbcname}
  \endfoot
    \caption{Paket-Sampling Varianten\label{tab:compositions:netflow-sampling}}
  \endlastfoot  
  Deterministisch & Jedes \textbf{N}-te Paket wird verwendet \\
  Zufällig & Alle \textbf{N}-Pakete wird eins ausgewählt und der Rest nicht
betrachtet\\
  Zeit basiert & Jede \textbf{N}-te Millisekunde wird das aktuellste Paket
ausgewählt
\end{longtable} 

Da ein Router (oder eine andere Probe) nicht mehr alle Pakete in Flows einteilen
muss wird dessen CPU entlastet und das Export Volumen zum Collector sinkt. Daher
eignet sich Paket Sampling besonders für Netze mit hohem Datenaufkommen, in
denen nicht alle Ströme erfasst werden müssen, sondern ein grober Überblick über
die Auslastung ausreicht.

\subsubsection{Templates}

Mit Version 9 von NetFlow wurden Templates eingeführt, durch die Flows flexibler
eingeteilt werden können als anhand der festen Menge an Flow-Keys in den vorigen
Versionen. IPFIX als Erweiterung von NetFlow V9 ermöglicht ebenfalls die
Verwendung von Templates. Dazu werden 2 unterschiedliche Arten von Datensätzen
von der Probe an den Collector geschickt: Template Records und Flow Data
Records.

Template Records definieren \texttt{\{Typ, Länge\}}-Paare, die angeben
welche Informationen in den Flow Data Records enthalten sind. Die Zuordnung der
Templates zu Flow Daten geschieht über eine Template ID, die im Template Record
enthalten ist und im Collector gespeichert wird.

Abbildung~\ref{fig:compositions:netflow-template} zeigt ein Export Paket, das
Template und Data Records enthält. Als \texttt{\{Typ, Länge\}}-Paar ist im
Template beispielsweise \texttt{\{IPv4\_SRCADDR (0x0008), Length = 4\}} zu
sehen, im Data Record erhält es dann einen konkreten Wert:
\texttt{192.168.1.12}. Über die ID -- in diesem Fall 256 -- werden die Daten dem
Template zugeordnet.

\begin{figure}
  \begin{center}
    \includegraphics[scale=0.7]{images/netflow-v9-template.png}
    \caption[Netflow FlowSet]{Netflow FlowSet, Quelle: Cisco NetFlow Services Solutions Guide,
    \url{http://www.cisco.com/en/US/docs/ios/solutions_docs/netflow/nfwhite.html\#wp1052564}, abgerufen am 14.11.2010}
    \label{fig:compositions:netflow-template}
  \end{center}
\end{figure}

Export Pakete die von der Probe an den Collector geschickt werden können
beliebig gemischte Template und Data Records enthalten
(siehe \cite[7, 8]{rfc-3954}):

\begin{itemize}
  \item Nur Template Records
  \begin{itemize}
    \item z.B. wenn die Probe gestartet wird und alle Templates and den
Collector schickt
  \end{itemize}
  \item Sowohl Template als auch Data Records
  \begin{itemize}
    \item z.B. wenn während dem verschicken von Data Records ein neues Template
an den Collector gesendet werden soll
  \end{itemize}
  \item Nur Data Records
  \begin{itemize}
    \item z.B. der Collector hat alle nötigen Templates und bekommt daher nur
Data Records geschickt
  \end{itemize}
\end{itemize}

\subsubsection{Probes}

Wie bereits erwähnt werden bei NetFlow Informationen über den IP-Verkehr durch
Probes gesammelt und an einen oder mehrere Collectoren gesendet. Probes können
Router sein die Hardware- oder Softwareseitig NetFlow unterstützen, aber auch
jeder beliebige andere Rechner in einem Netz, auf dem eine NetFlow fähige
Software (z.B. ntop) installiert ist. Dabei spielt es eine große Rolle an
welcher Position sich die Probe im Netz befindet.

Fungiert ein Router als Probe, an den mehrere Rechner als Stern-Topologie
angeschlossen sind und über den der Internet Zugang realisiert wird, wird dieser
Flow-Informationen über den gesamten Netzverkehr im LAN und \enquote{nach draußen}
erhalten. Ist die Probe hingegen auf einem Rechner installiert, der an einem
Router oder Switch angeschlossen ist, wird die Probe nur Informationen über den
Verkehr des Rechners selber erhalten. (Abgesehen von Broadcasts oder wenn der
IP-Verkehr des Switches/Routers über einen Mirror Port an den als Probe
fungierenden Rechner weitergeleitet wird.)

Der Transport der gesammelten Flow-Informationen geschieht bis NetFlow V8 über
UDP. V9 und IPFIX können auch andere Transportprotokolle verwenden, im RFC 3954
wird dazu SCTP genannt (\cite[6]{rfc-3954}), das im Gegensatz zu UDP
Überlastungskontrolle besitzt. Auch wenn es bei UDP keine Garantie dafür gibt,
dass Export Pakete ihr Ziel erreichen, enthält der NetFlow Header ab V5 eine
Squenz-Nummer, anhand der erkannt werden kann ob auf dem Weg von Probe zum
Collector Pakete verloren gegangen sind~\cite{netflow-cisco-sequence-number}.

\subsubsection{Collectoren}

Collectoren sind Software-Lösungen, die ihre Daten von NetFlow-Probes empfangen,
ggf. speichern und dem Benutzer in verschiedenen Arten präsentieren können. Im
Falle von \texttt{ntop} kann die Software  auch mittels NetFlow Plugin als Probe und
Collector in einem dienen. Die Aufbereitung hängt dabei davon ab, welche
Erkentnisse aus den Flow-Informationen gezogen werden sollen.

Abbildung~\ref{fig:compositions:netflow-live-action} zeigt eine Ansicht der
Cisco \enquote{LiveAction} Oberfläche. Diese stellt die Kommunikations-Beziehungen
zwischen mehreren Hosts eines LAN und der Außenwelt dar. Weiterhin sind die
Interfaces dargestellt (kleinere, grüne Kreise)  über die der Verkehr
stattfindet. Pfeile die von der unteren Kreis-Hälfte ausgehen bedeuten dabei
ausgehender Verkehr, Pfeile die zur oberen Hälfte führen Eingehender.

\begin{figure}
  \begin{center}
    \includegraphics[scale=0.3]{images/netflow-live-action-screen.jpg}
    \caption{LiveAction Screenshot, Quelle: CiscoGuard
    \label{fig:compositions:netflow-live-action}
(\url{http://www.ciscoguard.com/LiveAction.asp}, Abgerufen am 14.11.2010)}
  \end{center}
\end{figure}

\subsection{ntop \& nProbe}

Mithilfe von \texttt{ntop} und \texttt{nProbe} ist es Möglich NetFlow Daten zu sammeln und
darzustellen. Abbildung~\ref{fig:compositions:netflow-net} zeigt eine mögliche
Netztopologie mit folgenden Komponenten:

\begin{itemize}
  \item (1) Router, der Zugang zum Internet hat, sendet NetFlow Daten über einen
Mirror Port an (4)
  \item (2) Switch an dem 2 weitere Rechner und ein Server angeschlossen
sind
  \item (3) File-Server auf dem nProbe läuft, sendet NetFlow Daten an (4)
  \item (4) Workstation auf der ntop als Collector arbeitet
\end{itemize}

Während der Router die Verkehrsdaten ins Internet und zwischen dem linken und
rechtem Teilnetz auswerten kann, wäre Verkehr innerhalb des rechten Teilnetzes -
zum Beispiel wenn Workstation (4) Dateien vom File-Server (3) abruft für den
Router nicht sichtbar. Dies kann durch eine Probe auf dem File-Server (3)
umgangen werden, die ihrerseits NetFlow Daten sammelt und ebenfalls an den
Collector (4) schickt. Dadurch kann der gesamte IP-Verkehr des Netzes
ausgewertet werden.

\begin{figure}
  \begin{center}
    \includegraphics[scale=0.4]{images/netflow-net.png}
    \caption{Netzbeispiel}
    \label{fig:compositions:netflow-net}
  \end{center}
\end{figure}

\subsection{Verwendung in \f}

Wie im Überblick bereits angesprochen 
(siehe~\ref{sec:compositions:netflow-basics:overview}) könnte man NetFlow
im Kontext von \f dazu verwenden um mithilfe von Zeitreihenanalysen
(siehe Abschnitt~\ref{sec:compositions:timeanalysis}) Anomalien zu erkennen,
die auf Angriffe, Würmer, etc. hindeuten können. Neben der Ausarbeitung zu
Zeitreihenanalysen behandelt dies das \enquote{Netflow / IPFIX Security und Diagnose}
Referat. (Abschnitt~\ref{compositions:netflow-security})

Um NetFlow Daten zu erhalten könnte dazu ntop sowie nProbe verwendet werden.
Während ntop frei verfügbar ist, ist nProbe in der standard Version ab 100€
erhältlich (siehe \url{http://www.nmon.net/shop/cart.php}). Es besteht jedoch
die Möglichkeit für Universitäten eine freie Lizenz zu erhalten, solange diese
nicht für kommerzielle Zwecke genutzt wird, wodurch auch nProbe für bei \f
verwendet werden kann.

\subsection{Weiterführende Links \& Literatur}

\url{http://netflow.caligare.com/applications.htm} : Übersicht Software die
Netflow unterstützt. Abgerufen am 05.11.2010
