\section{CERT und CVE}
  \label{compositions:cert-cve}
  \authors{\LM}{}
\subsection{CERT}
\subsubsection{Einführung von CERT} Das erste Computer Emergency
Response Team (CERT) wurde im Anschluss an den Morris-Wurm 1988
gegründet. Der Morris-Wurm hatte damals ca. 10\% der Internet fähigen
Rechner infiziert und lahm gelegt. Das erste CERT wurde an der
Carnegie Mellon University, Pittsburgh gegründet und von der Defense
Advanced Research Projects Agency (DARPA/U.S.-Regierung)
unterstützt. Das Team sollte als Ansprechpartner bei zukünftigen
Security-Emergencies dienen und zusätzlich auftreten zukünftiger
Incidents verhindern.

Im Laufe der Zeit wurden immer mehr CERT-Organisationen, vor allem
nationale von Ländern unterstütze Organisationen gründet, um mit den
Anwachsenden des Internets und der Anzahl der Auftretenden Incidents
mit zuhalten.

Seit 1988 bzw. 1998 hat CERT damit begonnen sowohl Advisorys über
Schwachstellen, Beschreibung der Schwachstelle und mögliche Lösungen,
als auch Informationen über Security-Incidents, dies sind Viren,
Würmer oder allgm. Schädliche Programme, die Systeme Angreifen, zu
veröffentlichen. Seit 2004 werden diese Informationen von US-CERT im
Rahmen von Technical Cyber Security Alerts veröffentlicht.

\subsubsection{Advisories}

Ein Advisory beginnt mit der Beschreibung der betrachteten
Schwachstelle und auf welchen Systemen sie ausgenutzt werden
kann. Zusätzlich wird darauf eingegangen welche Auswirkungen das
Ausnutzen dieser Schwachstelle auf ein System haben kann.

Hauptsächlich behandelt ein Advisory die Möglichkeiten, die man zum
lösen der Schwachstellen besitzt, damit das System wieder sicher
ist. Hier bei Kann sich um Updates oder Upgrades von Anwendungen oder
Diensten oder bestimmten Konfigurationseinstellungen von System oder
Firewall handeln.

Zusätzlich wird noch auf Anmerkungen von Herstellern verwiesen, die
von diesem Problem betroffen sind.

\subsubsection{Incident Notes} Eine Incident Note bezieht sich auf
einzelne Security Incidents, meistens entweder ein Wurm oder Virus,
teilweise aber auch auf Code mit dem eine bestimmte Schwachstelle
ausgenutzt werden kann.

Eine Incident Note beginnt mit einer kurz Beschreibung des Incidents
gefolgt mit einer genaueren Beschreibung. Hierbei wird zum einen der
Verbreitungsweg beschrieben als auch die ausgenutzten
Schwachstellen. Außerdem wird beschrieben welche Schritte durchgeführt
werden, damit sich der Angreifer auf den System festsetzen kann.

Wenn es noch weiterführende Informationen von
Security-Information-Providern gibt wird auf diese anschließend
verlinkt.

Anschließend wird auf Möglichkeiten eingegangen die Infizierung zu
verhindern.  Dies umfasst zum einen aufspielen von vorhanden Patches,
aktuelle Versionen von Anti-Viren Programmen oder Verhaltensweisen
beim Erhalt noch Daten aus dem Internet.

Zusätzlich wird noch Hilfestellung gegeben, wie mit einem
kompromittierten System umgegangen werden kann.
\subsubsection{Aktuelle Arbeit} Mittlerweile veröffentlicht das erste
CERT keine Informationen mehr zu Vulnerabilities und Incidents sondern
beschäftigt sich hauptsächlich mit den folgenden Themen Bereichen:
\begin{itemize}
 \item Software Assurance
 \item Secure Systems
 \item Organizational Security
 \item Coordinating Response
 \item Training
\end{itemize}

Im Rahmen von Software Assurance befassen sie sich mit Möglichkeiten,
wie bei der Entwicklung von Anwendungen Schwachstellen vermieden
werden können. Cert bieten auch die Möglichkeit an, entwickelte oder
bereits verwendete Anwendungen auf Schwachstellen zu untersuchen um
Schwachstellen zu entdecken und beheben. Außerdem untersuchen sie auch
bösartigen Code um so einen besseren Einblick in die Funktionsweise zu
erhalten.

\subsection{US-CERT} Das US-CERT wurde 2003 als Teil des Department of
Homeland Security gegründet und dient als Ansprechpartner für
Security-Incidents für alle US-Behörden oder mit der US-Regierung
verbundenen Organisationen. Und dient auch als Ansprechpartner zum
Thema IT-Security und der US-Regierung.

US-CERT ist zwar unabhängig von anderen CERT-Organisationen, steht
aber sehr wohl in Kooperation mit diesen Organisationen um
beispielsweise Security Incidents zu beheben.

\subsubsection{Technical Cyber Security Alerts} Innerhalb der
Technical Cyber Security Alerts werden zeitnah Informationen über
Schwachstellen, Security Incidents und Exploits veröffentlicht. Dabei
handelt es sich um eine Mischung der Advisories und Incident Notes die
bis 2004 von CERT veröffentlicht wurden.

Sie bestehen aus folgenden Bestandteilen:\\
\begin{itemize}
 \item Befallene Systeme
 \item Beschreibung
 \item Auswirkung
 \item Gegenmaßnahmen
 \item Quellen
\end{itemize}

Genau wie bei den Advisories und Incident Notes wird die behandelte
Schwachstelle beschrieben genauso wie die Möglichkeiten diese
Auszunutzen und welche Systeme betroffen sein können.  Des weiteren
wird auch wieder die Auswirkungen und die Möglichen Gegenmaßnahmen
eingegangenen.

Für weiterführende Informationen wird auf
Security-Information-Provider und die Hersteller verwiesen.

\subsubsection{Cyber Security Bulletins} Bei den Cyber Security
Bulletins handelt es sich um eine wöchentliche Zusammenfassung
aktueller Schwachstellen die in der Woche registriert sind. Dabei
werden die Informationen aus der National Vulnerability Database (NVD)
zusammengefasst.

Die Schwachstellen werden anhand von CVE-Nummern beschrieben, und auch
auf ihren Risikofaktor anhand von CVSS eingestuft.

Für jede Schwachstelle wird aufgelistet welche Produkte und Hersteller
davon betroffen sind und dazu gibt es eine kurze Beschreibung. Neben
der Risikoeinschätzung anhand von CVSS wird noch auf den entsprechende
CVE oder Einträge von Security-Information-Providern verwiesen für
zusätzliche Informationen verwiesen.

\subsection{Vulnerability Notes Database}

Bisher haben wir uns mit verschiedenen Veröffentlichungen befasst, die
mit dem Thema Schwachstellen zu tun hatten. Diese bezogen sich alle
entweder nur auf bestimmte Security Incidents oder Vulnerabilities,
oder auf Vulnerabilities die in einem bestimmte Zeitraum auftraten.

Allerdings bergen nicht alle Vulnerabilities ein so hohen
Risikofaktor, dass eine Veröffentlichungen in einzelnen Berichten wie
Technical Cyber Security Alerts rechtfertigbar sind.  Risiko setzt
sich hierbei aus der Auswirkung der Schwachstelle zusammen,
z.B. Root-Rechte auf dem Host und der Häufigkeit von anfälligen
Systemen. Eine Schwachstelle die alle Windows-Versionen betrifft, hat
einen höheren Risikofaktor als eine Schwachstelle in einen veralteten
Ubuntu-Kernel.

Deswegen gibt es gesammelte Datenbanken die alle veröffentlichten
Informationen über Vulnerabilities sammeln und der Öffentlichkeit
Bereitstellen. Über diese können nun auch Vulnerabilities mit einem
geringeren Risikofaktor veröffentlicht werden.  Die gesammelten Daten
erleichtern es zusätzlich z.B. auch Systemadministratoren den
Überblick über Schwachstellen in eingesetzten Systemkonfigurationen zu
wahren.

Solche Datenbanken werden von verschiedenen
Security-Information-Providern, wie z.B US-CERT, wie z.B National
Vulnerability Database (NVD), bereitgestellt. Da leider die meisten
Security-Information-Provider auf eine eigene Namensgebung
zurückgreifen, gibt es leider teilweise nur geringe Vergleichbarkeit
zwischen den verschiedenen Datenbanken.

\subsection{CVE}
\subsubsection{Motivation} Um Einträge aus verschiedenen Datenbank
miteinander verknüpfen zu können, wurde 1999 Common Vulnerabilities
and Exposures (CVE) gestartet. Das Ziel war hierbei nicht, ein
Nachschlagewerk für alle vorhandenen Schwachstellen zu erstellen,
sondern eine Art Wörterbuch mit denen die verschiedenen Informationen
über Vulnerabilities nachgeschlagen werden können und nicht erst für
jeden Security-Information-Provider den Namen einer Vulnerability zu
suchen, da sie alle eigene Namenskonventionen verwenden.

Dies ermöglicht nun den Datenaustausch zwischen verschiedenen
Datenbanken und auch den Vergleich der Datenbanken. So kann
z.B. festgestellt werden welche Datenbanken Einträge zu bestimmten
Schwachstellen besitzen. Dadurch ergibt sich natürlich auch die
Möglichkeit Informationen über einen Vulnerability aus verschiedenen
Datenbanken zu gelangen.
\subsubsection{CVE-Nummer} Eine einzelne CVE-Nummer besteht aus den
Präfix CVE- gefolgt von der Jahreszahl und am Schluss einen
Identifier. Für den Identifier wird einfach die Anzahl der gefunden
Vulnerabilities in diesem Jahr genommen. Eine typische CVE-Nummer
sieht also folgendermaßen aus:\\
\begin{center} \textit{CVE-2008-4250}
\end{center} Mit dieser CVE-Nummer wird also die Vulnerability 4250
aus dem Jahr 2008 identifiziert.  Die CVE-Nummer wird beim einreichen
der Vulnerability bei einer \textit{CVE Candidate Numbering Authority}
vergeben. Gleichzeitig erhält der CVE auch den
\enquote{candidate}-Status und wird anschließend dem \textit{CVE
Editorial Board} vorgelegt. Wird der CVE akzeptiert, erhält er den
\enquote{entry}-Status, die Vergabe einer CVE-Nummer ist keine
Garantie dafür, dass die Vulnerability auch wirklich ein CVE-Entry
wird.
\subsection{Security Ecosystem}

Bisher haben wir nur die Möglichkeiten betrachtet, von denen
Informationen über Vulnerabilites erhalten werden können
betrachtet. Bisher wurde aber noch nicht der Zeitpunkt der
Veröffentlichung betrachtet. Der Veröffentlichungszeitpunkt ist
allerdings ein wichtiger Faktor der betrachtet werden muss.

Vorweg sollten nun einmal geklärt werden welche Personen in Rahmen von
Vulnerabilities-Informationen betrachtet werden:\\
\begin{itemize}
 \item Entdecker: Entdecker der Vulnerability, kann eine einzelne
Person oder Organisation sein
 \item Kriminelle: Versucht die Vulnerability auszunutzen um einen
eigenen Vorteil zu erhalten
 \item Hersteller: Versucht die Vulnerability zu beheben, da mit seine
hergestellte Software sicherer ist
 \item SIP: Stellt Informationen bereit, welche Schwachstellen es
gibt, welche Systeme betroffen sind wie die Schwachstelle behoben
werden kann
 \item Öffentlichkeit: Möchte aktuelle Informationen über
Schwachstellen erhalten, um das eigene Risiko abschätzen zu können
\end{itemize}

Wenn ein Entdecker seine Informationen an Kriminelle weiterverkauft,
werden diese versuchen die Schwachstelle immer noch geheim zuhalten,
damit diese weiter ausgenutzt werden kann und nicht vom Hersteller
behoben wird. Auch für die Hersteller ist es Interessant, die
Schwachstellen geheim zuhalten. Am liebsten würden diese die
Schwachstellen nicht beheben, da dieses für sie am billigsten
wäre. Andererseits müssen sie veröffentlichte Schwachstellen beheben,
da die Benutzer sonst Aufgrund von massiven Sicherheitsrisiken
aufhören würden ihre Software zu benutzen.

SIP kümmern sich darum, dass Vulnerabilitys veröffentlicht werden und
somit die Öffentlichkeit zu informieren. Der Zeitpunkt um eine
Vulnerability zu veröffentlichen muss gut gewählt werden. Denn zum
einen sollten die Hersteller die Möglichkeit erhalten, die
Schwachstellen rechtzeitig zu beheben, andererseits darf eine
Schwachstelle nicht zu lange verschwiegen werden.

Mit der Veröffentlichung einer Schwachstelle wird die Gefahr dieser
Schwachstelle massiv erhöht. Viele Kriminelle besitzen zwar nicht das
Know-How um Schwachstellen als erstes zu finden, sind aber sehr wohl
in der Lage aus den veröffentlichten Informationen der SIPs die
Schwachstelle herzuleiten und anschließend auszunutzen.


\subsection{Zusammenhang mit FIDIUS} Dadurch das wir nun die
Ausrichtung von \f geändert haben, ist das endgültige Ziel noch
nicht genau definiert. Im ersten Schritt wollen wir versuchen in Hosts
in einen Rechnernetz zu übernehmen. Hierzu müssen wir also die
verschiedenen Hosts analysieren und Schwachstellen zum eindringen
ausnutzen. Im zweiten Schritt sollen nun Informationen von IDS
verglichen werden, um die Trefferquoten vergleichen zu können.

Im ersten Schritt kann nun angesetzt werden. Die Vulnerabilities
Databases enthalten viele Schwachstellen und Informationen, welche
Systeme für diese anfällig sind.  Es gilt also, anhand von
Informationen die über ein Host gesammelt werden können, geeignete
Schwachstellen zu finden und diese dann anzuwenden.

Beim Anwenden von Schwachstellen kann zum Beispiel auf das
Metasploit-Framework zurückgegriffen werden. Dieses bietet die
Möglichkeit Exploits anhand von CVE-Nummern zu finden und anschließend
auszunutzen. CVE-Nummern sind hierbei die Ideale Verbindung zwischen
Vulnerability Notes Databases und Metasploit, da es ein Name für die
selbe Schwachstelle ist.

Ein geeigneter Anwendungsfall wäre als das finden von Schwachstellen
mittels einer Vulnerabilites Notes Database, z.B. NVD, und der
anschließenden Anwendung der Schwachstellen mit Metasploit.

 


