\section[Netflow \& IPFIX -- Security und Diagnose]
  {Netflow \& IPFIX\\Security und Diagnose}
\label{compositions:netflow-security}
\authors{\KUH}{}

\subsection{Einleitung}
Dieses Referat soll einen Überblick darüber geben, inwiefern sich
Netflow/IPFIX zur Diagnose und zur Verbesserung der Sicherheit eines
Netzes verwenden lassen. 

Durch die Verwendung von Netflow/IPFIX soll auch die Möglichkeit gegeben
werden, ein Netz mit hoher/sehr hoher Auslastung (GB-Bereich) zu
überwachen. Im Rahmen des Vortrages möchte ich vor allem darauf
eingehen, wie man mit Netflow/IPFIX es dem Benutzer (Administrator)
eines Netzes vereinfachen kann, Angriffe auf das Netz oder Anomalien im
Netz zu finden und zu beheben. Dabei gibt es verschiedene Verfahren, um
Schlüsse über mögliche Angriffe oder Anomalien zu ziehen.

Zum einen gibt es statische Verfahren, die ähnlich wie Snort auf
festgelegten Regeln basieren, zum anderen gibt es Verfahren, die
lernfähig sind und sich am verhalten des Traffics im Netz orientieren.

Als Beispiel für bisherige Implementationen dient die Programmsammlung
NFDUMP mit der Benutzeroberfläche NfSen.

\subsection{Netflow vs. libpcap}

In diesem Teil schaue ich mir Netflow und die libpcap ein wenig näher an.
Ich werde auf Tests der jeweiligen Implementierung eingehen und inwiefern
dies für den FIDIUS-Kontext von Bedeutung ist.

\subsubsection{libpcap}

Die libpcap wird bei vielen gängigen passiven Paketerfassungs-Programmen
verwendet. Die Bibliothek selber ist relativ einfach zu bedienen und es
gibt schon sehr gute Tutorials, die die Verwendung genauer erklären%
\footnote{\url{http://yuba.stanford.edu/~casado/pcap/section1.html},
abgerufen am 13.11.2010}.

Die libpcap wird bisher den Programmen: Tcpdump%
\footnote{\url{http://www.tcpdump.org/}, abgerufen am 14.11.2010}, 
Ethereal\footnote{\url{http://www.ethereal.com/}, abgerufen am 14.11.2010}
sowie das von uns bisher verwendete Snort verwendet.

Libpcap hat jedoch das Problem, dass das heutige Datenvolumen nicht mehr
vollständig erfasst werden kann. Wobei hier jedoch die Unterscheidung
zwischen großen und kleinen Paketen gemacht werden muss. Dies wird in
den Tests noch näher erläutert. Das eben genannte Problem wird auch
\textsl{livelock} genannt.

\begin{figure}
  \centering
  \includegraphics[width=\linewidth]{images/pcap_abb1}
  \caption{Darstellung des livelock Problems}
  \label{fig:netflow-security:livelock}
\end{figure}
 
\subsubsection{Netflow}

Netflow wurde von Cisco entwickelt um auch Netze mit höherem
Datenaufkommen, im GBit-Bereich, überwachen zu können. Näheres dazu im
Referat von Andreas Bender. Die Performance der Netflow Kollektoren skaliert stark mit der
eingesetzten Hardware. Dies wird in den Tests auch nochmal verdeutlicht.

\subsubsection{Tests}

Die Tests wurden einem Paper von Luca Deri entnommen\footnote{%
\url{http://citeseerx.ist.psu.edu/viewdoc/download?doi=10.1.1.58.3128&rep=rep1&type=pdf},
abgerufen am 14.11.2010}.

\paragraph{libpcap}

Der Test der libpcap entstand mit folgendem Sender und Empfänger:

\begin{itemize}
 \item Sender: Dual Core 1.8 GHz Athlon, Intel Gbit ethernet card
 \item Empfänger: Pentium III 550 MHz, Intel Gbit ethernet card
\end{itemize}

\begin{figure}
  \centering
  \includegraphics[width=.8\linewidth]{images/pcap_abb_2}
  \caption{Test der libpcap}
  \label{fig:netflow-security:test-libpcap}
\end{figure}

Wie man anhand der Tabelle erkennen kann, sind die prozentualen Werte
der Paket-Erfassung nicht wirklich gut. FreeBSD mit Polling bietet im
großen und ganzen die besten Werte im Gegensatz zu den beiden anderen
Betriebssystemen. Es fallen jedoch bei größeren Paketen noch viele
Pakete durchs Raster durch. Dies ist nicht sonderlich gut, wenn wir das
Netz überwachen wollen.

\paragraph{Netflow}

\subparagraph{Test 1}

Der erste Test mit Netflow entstand mit folgendem Sender und Empfänger:

\begin{itemize}
 \item Sender: Dual Core 1.8 GHz Athlon, Intel Gbit ethernet card
 \item Empfänger: Pentium III 550 MHz, Intel Gbit ethernet card
\end{itemize}

\begin{figure}
  \centering
  \includegraphics[width=.8\linewidth]{images/netflow_abb_1}
  \caption{Test von Netflow 1}
  \label{fig:netflow-security:test-netflow-1}
\end{figure}

In der Tabelle erkennt man, wieviele Pakete pro Sekunde verschickt
wurden. Dies kann man auch auf den Test der libpcap beziehen. Man erkennt
in der Tabelle deutlich, dass ohne Polling ein Großteil der kleinen
Pakete erfasst werden, jedoch die Rate der erfassten Pakete kleiner wird
bei größeren Paketen. Dies passiert mit Polling ebenso, nur werden
weniger kleine Pakete erfasst.

\subparagraph{Test 2}

Der zweite Test mit Netflow entstand mit folgendem Sender und Empfänger:

\begin{itemize}
  \item Sender: Dual Core 1.8 GHz Athlon, Intel Gbit ethernet card
  \item Empfänger: Pentium 4 1.7 GHz, Intel GE 32-bit
\end{itemize}

\begin{figure}
  \centering
  \includegraphics[width=.8\linewidth]{images/netflow_abb_2}
  \caption{Test von Netflow 2}
  \label{fig:netflow-security:test-netflow-2}
\end{figure}

Bei diesem Test wurde der Empfänger von der Hardware her um einiges
verbessert. Stellt jedoch noch keinen Highend Computer dar. Man erkennt
deutlich die Verbesserung im Gegensatz zum Ersten Test. Es werden immer
noch nicht alle kleinen Pakete erfasst, jedoch werden sämtliche größeren
Pakete erfasst. Somit erkennt man die Skalierung von Netflow mit der
Hardware sehr deutlich.

\paragraph{Ergebnisse der Tests}

Netflow liefert weitaus bessere Ergebnisse bei größeren Paketen ab.
Allerdings haben libpcap sowie auch netflow Probleme mit den kleinen
Paketen. Dies ist wohl daraus begründed, dass einfach zuviele Pakete
ankommen, die in einer Sekunde verarbeitet werden sollen. Wie man im
zweiten Test von Netflow gesehen hat, waren es über eine halbe Million
Pakete in der Sekunde. Da kam die Hardware an ihre Grenzen. 

Da die Tests jedoch auf relativ alten Computern ausgeführt wurden und
auch schon älteren Datums sind, stellen die Tests nicht den heutigen
Stand der Dinge dar. Aufgrund dessen, dass die meisten Angriffe mit
kleinen Paketen gemacht werden, bietet es sich weiterhin an mit den
aktuellen Sensoren weiter zu arbeiten.

\subsection{Probleme von Netflow}

Ebenso wie auch schon bei der libpcap, kann es bei Netflow zu einem
\textsl{livelock}~\ref{fig:netflow-security:livelock} kommen. Dazu hat
man bei Netflow jedoch mehrere Möglichkeiten diesen zu umgehen.

\begin{itemize}
  \item Verringern der Anzahl der Flows
  \item Flows  zusammenfassen
  \item Starten des filtern, bevor die Daten in den Flow gepackt werden
  \item Ändern der Filter-Parameter
  \item Ändern der Prozess Priorität
  \item Im schlechtesten Fall: Stoppen der Messung
\end{itemize}

Ein weiteres Problem bei Netflow stellt der Verlust von Flow-Daten dar.
Auch hierfür gibt es verschiedene Lösungsansätze um das Problem zu umgehen.

\begin{itemize}
  \item Erneutes senden der Flow-Daten
  \item Verbindungsabbruch und Fehlersuche
  \item Bestätigung der Flow-Daten vom Kollektor
\end{itemize}

Weiterhin können die Flow-Daten natürlich auch von Angreifern abgefangen
werden, sofern die Übertragung nicht anständig gesichert ist. Um so etwas
zu umgehen, sollten zum einen die Daten nicht übers Internet versendet
werden. Die Daten sollten Anonymisiert werden, sowie für die Übertragung
verschlüsselt werden.

\subsection{Analyse Verfahren}

Zum Analysieren von Flow-Daten gibt es zwei gängige Verfahren. Zum einen
Top-N-Analyse und zum anderen die Analyse über Zeitreihen. 
Näheres über Zeitreihen ist dem Referat von Daniel Kohlsdorf zu entnehmen.

\subsubsection{Top-N-Analyse}

Bei der Top-N-Analyse werden die Flow-Daten anhand fester Regeln, ähnlich
wie bei Snort, durchgesehen. Dabei hat der Benutzer mehrere Möglichkeiten
diese Regeln selber zu definieren. Es sind bei weitem nicht so viele
Regeln wie z.B. in Snort. Die meisten Kriterien für die Regeln sind
gespeicherte Daten im flow. Ausserdem werden viele Regeln schon von
Visualisierungsprogrammen zusammengefasst. Somit brauch der Benutzer
z.B. nur nach den Elementen suchen, wo der meiste Traffic anfällt. 

\subsection{NFDUMP}

Für das Referat habe ich mir ein Programm-Paket angeschaut, dass schon
die Verarbeitung von Netflow-Daten anbietet. NFDUMP\footnote{%
\url{http://nfdump.sourceforge.net/}, abgerufen am 14.11.2010} besteht
aus folgenden einzelnen Programmen.

\begin{itemize}
  \item nfcapd: nfcapd ist ein Programm um die Netflow-Daten aus dem Netz
  zu bekommen.
  \item nfdump: Dient der Analyse der Netflow-Daten.
  \item nfprofile: Aufteilen der Netflow-Daten anhand von vorher
  definierten Profilen.
  \item nfclean.pl: Löscht alte, nicht mehr benötigte Netflow-Daten.
  \item ft2nfdump: Ein Tool, dass die Daten von dem Programm flow-tools
  konvertiert.
\end{itemize}

Ein weiterer Vorteil den die Ansammlung bietet, ist das die Netflow-Daten
schon über ein Flag direkt anonymisiert werden können. Somit erspart man
sich ein eigenes Anonymisierungstool.

\subsection{NfSen}

NfSen\footnote{\url{http://nfsen.sourceforge.net/}, abgerufen am 14.11.2010}
ist die Benutzeroberfläche für das Programm-Paket NFDUMP. Mit diesem
Frontend-Programm werden die Netflow-Daten grafisch für den Benutzer 
aufbereitet. Dabei werden verschiedene Möglichkeiten für die Ansicht
angeboten. 

\begin{figure}
  \centering
  \includegraphics[width=\linewidth]{images/nfsen_abb_1}
  \caption{Allgemeine Ansicht}
  \label{fig:netflow-security:test-nfsen-1}
\end{figure}

\begin{figure}
  \centering
  \includegraphics[width=\linewidth]{images/nfsen_abb_2}
  \caption{Genauere Ansicht eines Flows}
  \label{fig:netflow-security:test-nfsen-2}
\end{figure}

\begin{figure}
  \centering
  \includegraphics[width=\linewidth]{images/nfsen_abb_3}
  \caption{Noch genauere Ansicht}
  \label{fig:netflow-security:test-nfsen-3}
\end{figure}

Ausserdem besteht die Möglichkeit eigene Plugins zu entwickeln und zu
integrieren. 

\subsection{Verwendung in FIDIUS}

Der Nutzen für das Projekt wäre wohl, dass man sich bei Angriffen
anschaut, welche vom jeweiligen Sensor erkannt werden. Sei es nun Snort
oder ein Netflow Kollektor. Meiner Meinung nach sollten wir wohl beide
Sensoren benutzen, um den Großteil der Angriffe mitbekommen zu können.

\subsection{Weiterführende Links \& Literatur}

\begin{itemize}
  \item \url{http://www.splintered.net/sw/flow-tools/}
  \item \url{http://eprints.eemcs.utwente.nl/16911/01/JNSM-Netflow.pdf}
  \item \url{http://citeseerx.ist.psu.edu/viewdoc/download?doi=10.1.1.60.6198&rep=rep1&type=pdf}
  \item \url{http://citeseerx.ist.psu.edu/viewdoc/download?doi=10.1.1.74.6338&rep=rep1&type=pdf}
\end{itemize}
