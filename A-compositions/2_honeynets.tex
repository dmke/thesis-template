\section{Honeynets}
\label{compositions:honeynets}
Ziel dieser Arbeit ist es kurz zu erläutern was
ein Honeynet ist, wie es funktioniert, wozu es dient und ob der
Einsatz innerhalb des Projektes FIDIUS lohnenswert ist.

\subsection{Honeypot} Ein Honeypot ist ein für einen Angreifer
interessant wirkendes System in einem Produktivnetz. Dabei ist der
Honeypot stark überwacht und meldet jegliche Interaktion, die sich mit
oder von ihm ergibt. In einem Produktivnetz dient ein Honeypot als
Frühwarnsystem, da er sich, im Vergleich zu den restlichen Komponenten
im Produktivnetz, eher als leichtes aber dennoch interessantes Ziel
darstellt. Ein möglicher Angreifer soll durch einen Honeypot vom
Restnetz abgelenkt werden, und seine Angriffe auf den Honeypot
richten.

Ein Honeypot zeichnet sich dadurch aus, dass er so gut wie keine
\textit{False-Positives} erzeugt, da es im Netz ohne Angriff keine
Interaktion mit dem Honeypot gibt. Allerdings bringt ein Honeypot auch
Gefahren mit sich, denn ein Angreifer könnte im Stande sein ihn als
Angriffsplatform zu nutzen.  Deswegen ist es besonders wichtig einen
Honeypot dauerhaft zu überwachen und ihn so einzurichten, dass er im
Zweifelsfall den Dienst verweigert.

\subsection{Honeynet} Ein Honeynet ist ein Netz von verschiedenen
Honeypots, erstellt um sie durch die \textit{black-hat community}
angreifen zu lassen. Im Gegensatz zu einzelnen Honeypots im Netz die
die Aufmerksamkeit von Hackern auf sich ziehen sollen, dient ein
Honeynet dazu Taktiken und die Vorgehensweise von Hackern zu erlernen
und zu analysieren.  Ein Honeynet ist so konstruiert, dass es zwar
übernommen werden kann und man so die Schritte eines Angreifers
nachvollziehen kann, aber das Honeynet selbst nicht weiter benutzt
werden kann um Angriffe außerhalb dieses Netzes durchzuführen.  Um
sicherstellen zu können, dass auftretende Logereignisse nicht vom
Angreifer zerstört werden, ist es sinnvoll für diesen Zweck einen
\textit{Remote-Log-Server} einzurichten. Ein weiterer Vorteil eines
\textit{Remote-Log-Servers} ist es, dass es einem die Möglichkeit
bietet alle Logereignisse an einer zentralen Stelle zu prüfen.

Die Einsatzmöglichkeiten von Honeynetzen variieren von einem reinen
Forschungsnetz bis hin zum Einsatz in einem Produktivnetz. Dabei muss
man sich bewusst sein, dass ein Honeynet durchaus im Stande ist die
Sicherheit eines Netzes, ähnlich wie ein Honeypot, zu erhöhen, aber es
auch einige Risiken mit sich bringt.

Vor dem Aufsetzen eines Honeynet ist es wichtig die folgenden Punkte
zu beachten:
\begin{itemize}
  \item Die Komplexität des Netzes erhöht sich.
  \item Der Zeitaufwand / Kosten für die Wartung und Analyse des
Honeynetzes ist erheblich.
  \item Es ergeben sich rechtliche Verpflichtungen bezüglich des
Verfahren und der Haftbarkeit bei der Übernahme des Honeynetzes.
  \item Die Sicherheit des Produktivnetzes kann sich bei einer
Übernahme verringern.
\end{itemize} 

Grundsätzlich muss man vor der Einrichtung eines
Honeynetzes den Kosten / Nutzen Faktor abschätzen.

\subsection{Funktionsweise und Aufbau} Dieser Abschnitt befasst sich
mit den nötigen Schritten ein Honeynet einzurichten und wie ein
Honeynet im Allgemeinen aufgebaut ist. Die hier aufgeführten Aspekte
beruhen auf den vom \textit{Honeynet-Project} empfohlenen Schritten
ein Honeynet mit Hilfe der \textit{Honeywall} einzurichten.

Die \textit{Honeywall} wurde vom \textit{Honeynet-Project} entwickelt
und ist ein installierbares \textit{Layer-3-Gateway} basierend auf
einem minimal gehaltenen Fedora OS. Der Funktionsumfang der
\textit{Honeywall} wird in dem jeweiligen Abschnitt näher erläutert.

\subsubsection{Data Control} Sobald ein System im Honeynet übernommen
wurde, ist es wichtig die Möglichkeiten des Angreifers unbemerkt so zu
Kontrollieren, dass das übernommene System keinen Schaden an
Fremdsystemen anrichten kann. Diese Kontrollmechanismen fasst man in
diesem Zusammenhang unter dem Begriff \textit{Data Control}
zusammen. Bei Einrichtung eines Honeynetzes, ist dies der wichtigste
zu beachtende Schritt.

Die \textit{Honeywall} bietet für diesen Teil mehrere redundante
Funktion zu Realisierung des \textit{Data Control}. Zum einen kommt
die \textit{Honeywall} mit einem vorkonfigurierten Network Intrusion
Prevention System (NIPS), dass den ausgehenden Datenverkehr überwacht
und bekannte Angriffe nach außen verhindert. Zum anderen ist die
\textit{Honeywall} im Stande ausgehende Verbindungen zu Zählen und zu
limitieren. Der gesammte ausgehende Verkehr wird mittels
\textit{Ip-Tables} zum NIPS geleitet und sorgt dafür, dass wenn das
NIPS ausfällt garkein Verkehr mehr nach außen gelangt.


\subsubsection{Data Capture} Ein Honeynet ergibt erst Sinn, wenn es
möglich ist die Anfallenden Aktivitäten im Honeynet zu analysieren. Zu
diesem Zweck ist es Notwendig Verfahren zu Implementieren die auf so
vielen Ebenen wie möglich das Verhalten eines Angreifers
aufzeichnen. An dieser Stelle ist nochmals zu erwähnen, dass die
anfallenden Logereignisse in einem \textit{Remote-Log-Server}
zusammenlaufen sollten, da es die Analyse vereinfacht und
sichergestellt wird, dass der Angreifer nicht in der Lage ist
Logereignisse zu verändern.  Es ist empfehlenswert Logs zur Auswertung
in folgenden Bereichen bereitzustellen:
\begin{itemize}
  \item Firewall Logs
  \item Network Traffic
  \item System Activity
\end{itemize}
\paragraph*{Firewall Logs} Dieser Teil wird bereits durch die
\textit{Ip-Tables} realisiert, da es alle ein und ausgehenden
Verbindungen logt.
\paragraph*{Network Traffic} Für diesen Zweck lauscht ein Snort
Intrusion Detection System auf dem inneren Netzinterface der
\textit{Honeywall}. Diese Aufgabe könnte zwar auch das bereits
erwähnte NIPS übernehmen, doch ist es sicherer redundant zu arbeiten.
\paragraph*{System Activity} Das loggen der \textit{System Activity}
kann durch \textit{Debugger} oder auch durch das Programm
\textit{Sebek} übernommen werden. Hierzu sendet zum Beispiel
\textit{Sebek} alle auf einem System auftretenden Ereignisse per UDP
an die \textit{Honeywall} oder einen \textit{Remote-Log-Server}.

\subsubsection{Alerting} Auch wenn die Gefahr, dass das Honeynet
missbraucht wird, durch \textit{Data Control} vermindert wird, ist es
weiterhin nötig bei einem Angriff Zeitnah zu reagieren. Aus diesem
Grund muss vor der Endgültigen Inbetriebnahme des Honeynetzes
sichergestellt sein, dass Angriffe gemeldet werden. Wie dies Umgesetzt
wird, ist allerdings von Fall zu Fall unterschiedlich und soll hier
nicht näher erläutert werden.  \clearpage

\subsubsection{Architektur} Der typische Aufbau eines Honeynetzes
(siehe Abb.~\ref{fig:honeynet}) besteht aus der Trennung vom
Produktivnetz (ähnlich einer DMZ) und dem Honeynet mit Hilfe der
\textit{Honeywall}
\begin{figure}[htbp] 
    \center 
    \pgfimage[interpolate=true,width=1.0
    \linewidth]{images/honeynet.png}
    \caption{Honeynet Architecture [KYE: Honeynets 2006]}
    \label{fig:honeynet}
\end{figure}

\subsection{Einsatz in FIDIUS} Der Einsatz eines Honeynetzes im
Projektkontext kann in zwei Arten umgesetzt werden.
\paragraph*{Als typisches Honeynet} Unter typisches Honeynet wird in
diesem Kontext davon ausgegangen, dass das Honeynet einen Zugang zum
Internet bekommt und somit von außen erreichbar ist.  Es würden also
nicht nur Projektmitglieder Zugriff auf das System erhalten, sondern
auch Angreifer aus dem Internet. Dieses Vorgehen hätte den Vorteil,
dass eine große Anzahl von Angriffen auf das System erfolgt und sehr
viele Daten über Angriffe gesammelt werden könnten. Weiterhin wären
die Angriffsdaten sehr realistisch und es bestünde die Möglichkeit
neue Angriffe zu identifizieren. Allerdings würde dieses Vorgehen
einen stark erhöhten Aufwand mit sich bringen, da die aktuellen
Fähigkeiten des Projektes noch nicht ausreichen, um die anfallenden
Daten zu analysieren. Der Aufwand beschränkt sich zu dem nicht nur auf
das erlernen der benötigten Fähigkeiten, sondern auch auf die
Umsetzung der in dieser Arbeit beschrieben Aspekte bezüglich der
Einrichtung des Honeynet. Damit müsste FIDIUS die rechtliche Situation
klären und ein Konzept erarbeiten, dass es erlaubt im Falle einer
Übernahme angemessen zu reagieren.

\paragraph*{Als Sandbox} Der Einsatz eines Honeynetzes als Sandbox
könnte auch von Interesse im Projekt sein, denn ein Honeynet ist von
sich aus sehr realistisch Aufgebaut und deswegen ein ideales
Testnetz. Zudem würde es die Möglichkeit bieten, die ausgeführten
Angriffe näher zu betrachten und zu analysieren. Es könnte also nicht
nur beobachtet werden wie ein Intrusion Detection System auf Angriffe
reagiert, sondern auch wie der Angriff sich auf ein System
auswirkt. Die durch die Honeynet Architektur zusätzlich gewonnen Daten
könnten so verwendet werden, dass die im Projektkontext benutzen
System verbessert werden.  Im Vergleich zum typischen Honeynet, hätte
dieses Vorgehen den Vorteil, dass die ausgeführten Angriffe bekannt
sind. Es würde also entfallen, dass man die Angriffe erst einmal
identifizieren müsste. Leider ist dieser Vorteil auch gleichzeitig ein
Nachteil, denn FIDIUS würde so nur Angriffe analysieren können die dem
Projekt bereits bekannt sind.

\subsection{Fazit} An sich ist der Einsatz eines Honeynetzes im
Projektkontext wünschenswert, da es viel Potential bietet
Angriffsmethoden zu erlernen und zu studieren. Ebenso kann ein
Honeynet als Sandbox die Gefahr verringern, dass der noch zu
entwickelnde FIDIUS-Wurm aus dem Testnetz entwischt.

Für ein typisches Honeynet besitzt das Projekt allerdings nicht die
zeitlichen Ressourcen, da man davon ausgehen muss, dass für jede
Stunde Aktivität im Honeynet ca. 40 Personenstunden benötigt werden
sie zu analysieren.

Für den Einsatz als Sandbox ist es auch zweifelhaft in wiefern sich
der Kosten / Zeitaufwand mit dem Nutzen deckt.
