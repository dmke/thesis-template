\section{Metasploits Meterpreter}
\label{compositions:meterpreter}
\authors{\HM}{}

\subsection{Einleitung}

Meterpreter, Kurzform für Meta-Interpreter, ist ein hochentwickelter
Payload für Metasploit \cite{skape-meterpreter}. Metasploit ist ein
erweiterbares Framework zum Penetration Testing, das vielen Exploits
und damit kombinierbaren Payloads mitliefert (siehe~\ref{compositions:msf}).
Meterpreter ist ein Payload der von einem Exploit in Metasploit
ausgeführt werden kann. Es ist mit vielen Exploits von Metasploit
kompatibel und lässt sich anstatt einer Systemshell wie z.B.
\texttt{/bin/bash} oder \texttt{cmd.exe} wählen, die Bedienung von
Meterpreter ist für den Angreifer ähnlich zu einer Systemshell.

\subsection{Probleme von System Shells}
\label{compositions:meterpreter:systemshell}

Wenn ein System angegriffen wird soll dieses normalerweise sehr leise
geschehen, damit installierte Sicherheitssysteme wie z.B. \acr{nids},
\acr{hids} und Antivirenprogramme dieses nicht bemerken. Die Benutzung
von Systemshells als Payload von Exploits bereitet bei Angriffen auf
Systemen, wenn mehr als nur die Angreifbarkeit durch einen bestimmten
Exploit gezeigt werden soll, Probleme.

Wenn die Systemshell als Payload eines Exploits von diesem gestartet
wird, muss dazu ein neuer Prozess erzeugt werden oder der alte Prozess
komplett überschrieben werden z.B. mithilfe von \texttt{exec()}.
Dieses Verhalten ist sehr auffällig, vor allem wenn ein Programm welches
sonst nie eine Shell starten würde dieses auf einmal tut.
Viele \acr{hids} und Antivirenprogramme haben eine Heuristik, um die
Wahrscheinlichkeit für einen Angriff zu ermitteln. Ein Aufruf von
\texttt{exec()} oder ähnlichen Funktionen geht immer über einen
Systemcall in den Kernel und lässt sich dort sehr gut abfangen und trägt
dann stark dazu bei die Aktion als Angriff zu bewerten.

Vor allem in Unix Systemen werden Prozesse in \texttt{chroot}-Umgebungen
ausgeführt und auf Desktopsystemen werden immer mehr Client Programme,
wie Internet Browser, innerhalb einer Sandbox ausgeführt. In diesen
Umgebungen ist nicht immer ein Zugriff auf die Systemshell möglich,
z.B. weil der Prozess, der in der \texttt{chroot} läuft diese nicht
benötigt.

Für bestimmte Aktionen die nach dem Angriff eines System ausgeführt
werden sollen, wie z.B. das Sammeln von Informationen werden zusätzliche
Tools benötigt. Diese Tools müssen manuell nachgeladen werden, wenn sie
noch nicht auf dem angegriffenen System installiert sind, was
zusätzliche, auffällige Aktionen verursacht. Diese Programme würden z.B.
über HTTP auf die Festplatte des angegriffenen Systems geladen und dann
mithilfe der Shell ausgeführt. Daten die auf die Festplatte geschrieben
werden, werden aber normalerweise vorher vom Antivierenprogramm
analysiert und der Schreibvorgang wird unterbunden, falls er als Teil
eines Angriffs erkannt wird. Lese- und Schreiboperation die sich nur im
Arbeitsspeicher abspielen werden von Antivirenprogrammen meistens nicht
untersucht, da dieses viel schwieriger zu implementieren wäre und die
Systemperformance zu stark beeinträchtigen würde. Die Überprüfung von
Lese- und Schreiboperationen auf Festplatten stellt kein großes
Performance-Problem da, weil dieses Medium im Gegensatz zum
Arbeitsspeicher sehr langsam ist und deswegen auch nicht so viele
Operationen darauf ausgeführt werden.

Eine Systemshell wie \texttt{cmd.exe} bietet auch oft nicht den vom
Angreifer bevorzugten Bedienkomfort oder ist nicht so einfach skriptbar.
Desweiteren hat jede Systemplattform eine etwas andere Shell, wobei sich
Skripte, die für die Windows-Shell \texttt{cmd.exe} geschrieben wurde
nicht auf Unix-Systemen mit einer Bash nutzen lasen und umgekehrt. Damit
müssten Skripte, die bestimmte Aktionen nach einem Angriff ausführen
zumindest für die beiden großen Plattformen separat erstellt und gepflegt
werden.

Ein Vorteil von Systemshells ist, dass sie schon auf dem angegriffenen
System installiert sind und damit nicht extra übertragen werden müssen,
was Bandbreite spart und vielleicht auch dem \acr{nids} weniger
Möglichkeiten zum Erkennen gibt. Teilweise darf eine Payload bei einem
Exploit eine bestimmte Größe nicht überschreiten, was für den Aufruf der
Systemshell meistens kein Problem darstellt, aber bei eigener Software,
die größer ist Probleme bereiten kann.

\subsection{Meterpreter}
\label{compositions:meterpreter:meterpreter}

Meterpreter versucht fast alle die Probleme, die bei der Benutzung von
Systemshells auftreten zu beheben. Dazu ist Meterpreter so gebaut, dass
es zur Laufzeit weiteren Code nachladen und ausführen kann, je nachdem
was gerade benötigt wird. Dadurch muss zu Anfang nur ein kleines
Programm übertragen werden welches dann die restlichen Funktionen
nachlädt.

Der Meterpreter ist in zwei Teile aufgeteilt: Der Server, der in C
geschrieben ist, läuft auf dem angegriffenen Host und der Client, der in
Ruby geschrieben ist, läuft beim Angreifer. Diese beiden Komponenten
kommunizieren miteinander über ein speziell für den Meterpreter
entwickeltes Protokoll, welches über verschiedene standardisierte
Protokolle wie TCP, UDP und HTTP gelegt werden kann.

Der Angreifer kann den Meterpreter über eine Shell bedienen, die auch
\emph{tab completion} und eine \emph{command history} bietet. Meterpreter
ist mithilfe von Ruby komplett skriptbar. Es gibt viele Skripte, die
schon mit Metasploit ausgeliefert werden und nur gestartet werden müssen,
es ist aber auch relativ einfach eigene Skripte zu schreiben um noch
fehlende Funktionen nachzurüsten.

\subsection{Funktionen}
\label{compositions:meterpreter:funktionen}
Meterpreter bringt schon viele nützliche Funktionen mit, die man bei
einer Systemshell nur mithilfe von externe Programmen erreichen kann.
Es ist möglich Dateien vom angegriffenen System runter- und auf das
System hochzuladen, direkt über die Meterpreter Verbindung. So muss keine
Extra-Verbindung aufgebaut werden oder ein Extra-Programm wie wget
installiert werden.

Mithilfe von Port-Forwarding können einzelne Port vom System des
Angreifers auf den angegriffenen Host geleitet werden. Dieses bietet
ähnliche Funktionen wie das SSH-Port-forwarding. Die Daten werden über
die Meterpreter Verbindung geleitet und können wie die anderen Daten
verschlüsselt werden. Es ist auch möglich einen Socks4a Proxy ähnlich zu
\cite{rfc-1928} über diese Verbindung laufen zu lassen. Weitere Angriffe
aus Metasploit können auch über diese Verbindung geleitet werden indem
in der Metasploit Konsole mithilfe von route alle Pakete, die an Hosts
in einem bestimmten Subnetz gerichtet sind, über diese Verbindung
geroutet werden. Zum Betrachten und Verändern der Routingtabelle kann
der Befehl \texttt{route} verwendet werden.

Zum Abhören von Netzverkehr kann die Meterpreter Sniffer-Erweiterung
verwendet werden. Diese Erweiterung verwendet die libpcap\footnote{%
\url{http://www.tcpdump.org/}} und sendet alle abgefangenen Pakete an
den Angreifer, damit dieser sie auswerten kann und weitere Angriffe
planen und ausführen kann.

Die Tastatureingaben des Nutzers können mithilfe von \texttt{keyscan}
abgefangen werden und mit \texttt{webcam} ist es sogar möglich, wenn
eine Webcam an dem Rechner vorhanden ist, mit dieser Bilder zu machen.
Zum verwischen der Spuren kann \verb!clearev! verwendet werden.
Dieses Tool löscht den Windows Event Log.

Die Erweiterung Railgun eignet sich dafür, um mit Ruby-Skripten
Funktionen auf Windows System Bibliotheken aufzurufen. Es gibt schon für
folgende DLLs vorgefertigte Definitionen zum Aufrufen von Funktionen:
\textit{advapi32.dll}, \textit{iphlpapi.dll}, \textit{kernel32.dll},
\textit{ntdll.dll}, \textit{shell32.dll}, \textit{user32.dll} und
\textit{ws2\_32.dll}. Damit ist es dann z.B. möglich ein Fenster auf dem
Desktop anzuzeigen oder Komplexere Aktionen durchzuführen.

Der übertragene Programmcode von Meterpreter kann auch mithilfe von
Encodern versteckt werden, damit die Signaturerkennung bei \acr{nids},
\acr{hids} und Antivirenprogramme erschwert wird. Dazu wird der
eigentliche Programmcode des Meterpreters verschlüsselt und an den
Anfang wird ein Decoder mit den Schlüssel zum Entschlüsseln geschrieben.
Wenn Meterpreter gestartet wird, wird zuerst der Decoder gestartet der
mit dem mitgelieferten Schlüssel den Meterpreter entschlüsselt und
startet. Zur Verschlüsselung reicht oft schon eine XOR Verknüpfung, da
der Schlüssel sowieso mitgeliefert wird und der Code nur jedes mal immer
anders aussehen soll. Ein Decoder für eine XOR-Verschlüsselung ist auch
sehr einfach und klein zu implementierten.

\subsection{Meterpreter Skripte}
\label{compositions:meterpreter:meterpreter-skripte}

Die Meterpreter Skripte laufen auf dem Client, dem System des Angreifers,
und sagen dem Server welche Operationen er ausführen soll. Der Server,
der auf dem angegriffenen Host läuft, ist nicht direkt skriptbar, sondern
nur indirekt über den Client. Es muss also immer eine Verbindung zwischen
Client und Server bestehen während die Skripte ausgeführt werden.

Meterpreter bringt schon einige Skripte mit, die nützliche Aufgaben
erledigen.

\begin{longtable}{lp{.7\linewidth}}
  \rowcolor{Beige}
    Name & Beschreibung \\
  \endhead
    \caption[]{Meterpreter Skripte\\\tabelletbcname}
  \endfoot
    \caption{Meterpreter Skripte\label{tab:compositions:meterpreter-scripts}}
  \endlastfoot
  getgui & Windows Remote Desktop aktivieren und Benutzer dafür anlegen \\
  hashdump & Windows Passwort Hashes anzeigen \\
  arp\_scanner & ARP Scanner \\
  enum\_firefox & Firefox Einstellungen (History, Passwörter, ... )
    kopieren (gibt es auch für andere Browser)\\
  checkvm & Überprüft, ob es sich um eine VM handelt \\
  getcountermeasure & Überprüft welche Schutzmechanismen das System hat \\
  killav & Versucht die aktiven Antivirenprogramme auszuschalten \\
  hostsedit & Bearbeiten der /etc/hosts Datei
\end{longtable} 

\subsection{Architektur}

\subsubsection{Anforderungen}

Um die Probleme die bei der Benutzung von Systemshells auftreten
(siehe~\ref{compositions:meterpreter:systemshell}) zu umgehen, wurden
beim Entwurf bestimmte Anforderungen an die Architektur des Meterpreter
gestellt \cite{skape-meterpreter}. Es muss kein neuer Prozess gestartet
werden, um Meterpreter zu starten, sondern Meterpreter muss in dem
Prozess in dem der Exploit ausgeführt wurde weiter laufen können ohne
den kompletten Inhalt auszutauschen. Dadurch kann sich Meterpreter
besser vor \acr{hids} und Antivirenprogrammen verstecken. Zusätzlich
soll es auch in einer \texttt{chroot}-Umgebung laufen können und keine
speziellen Anforderungen an installierte Programme auf dem System stellen.
Meterpreter soll sich zusätzlich dynamisch zur Laufzeit erweitern lassen.

\subsubsection{Aufbau}

Wie schon in~\ref{compositions:meterpreter:meterpreter} beschrieben, ist
der Meterpreter eine Client-Server Anwendung. Die Kommunikation der
beiden Teile miteinander findet über den vom Stager aufgebauten Socket
statt. Ein Stager ist ein kleines Programm welches direkt an den Exploit
als Payload angehängt wird. Oft können direkt mit dem Exploit nur sehr
kleine Payloads mit übertragen werden, weil der Exploit nicht viel Platz
für den Payload bietet. Um trotzdem größere Anwendungen zu verwenden
wird direkt im Payload des Exploits nur ein kleiner Stager mitgeliefert,
der dann später den eigentlichen Payload, z.B. Meterpreter nachlädt und
startet.

Dabei baut der Stager schon eine Socket-Verbindung auf, die Meterpreter
dann später für seine Kommunikation weiterbenutzt. Es gibt Stagers für
verschiedene Protokolle, unter anderem für (TCP und UDP jeweils
\textit{reverse}\footnote{angegriffener Host baut Verbindung zu
Angreifer auf} und \textit{bind}\footnote{angegriffener Host öffnet Port
und Angreifer verbindet sich}). Zusätzlich gibt es auch einen Stager,
der eine HTTP- oder HTTPS-Verbindung aufbauen kann und somit die
Kommunikation gut tarnen kann. Eine Verbindung über TCP, UDP und HTTP
kann mithilfe von SSL verschlüsselt werden, wobei die Validität der
Zertifikate nicht überprüft wird. Die Überprüfung der Zertifikate würde
die Sicherheit nicht stark erhöhen, da der private Schlüssel des
Meterpreter-Servers und das Zertifikat des Clients mit dem Exploit
übertragen werden müssten und diese Daten abhängig vom Exploit auch
schon unverschlüsselt sind und manipuliert werden können.

\paragraph{Server}

Der Server läuft auf dem angegriffenen Host und ist in der
Programmiersprache C geschrieben, zum Kompilieren wird Microsoft Visual
Studio 2008 oder 2010 benötigt. Da der Meterpreter hauptsächlich dazu
entwickelt wird Windows-Systeme anzugreifen, stehen auch nur unter
diesem System alle Funktionen zur Verfügung. Zusätzlich wird auch
Linux unterstützt, für Mac OS X gibt es noch keinen offiziellen Server,
es gibt aber schon einige Anfänge dazu, wobei diese sich eher an das
iPhone richten, welches sehr ähnlich zu Mac OS X ist\dots

Für PHP-Webanwendungen gibt es auch einen Meterpreter, der in PHP
geschrieben ist und sich als Payload für Exploits von PHP -Anwendungen
eignet. Für Java-Applets und andere Java-Anwendungen gibt es auch noch
eine spezielle Java-Version des Meterpreters.

Diese verschiedenen Versionen von Meterpreter unter anderem für Java und
PHP werden benötigt, da Meterpreter im Kontext des angegriffenen
Prozesses laufen soll. Zum Starten von Meterpreter wird
Remote-Library-Injection verwendet, also der Meterpreter-Code wird in
den laufenden Prozess geschrieben und dann ausgeführt. Deswegen müssen
keine Daten auf die Festplatte geschrieben werden und es muss auch kein
neuer Prozess erstellt werden, was Schutzprogramme alarmieren könnte.
Nachdem der Meterpreter in den Prozess geladen wurde funktioniert der
ursprüngliche Prozess meistens  nicht mehr, dieses hängt aber vom
Exploit ab.

Der Meterpreter-Server kann auch, nachdem er in einem Prozess gestartet
wurde, in einen anderen Prozess wechseln. Das nennt sich Migration und
ist z.B. Sinnvoll, wenn der angegriffene Prozess beendet werden könnte
oder der Angreifer für bestimmte Aktionen in einem anderen Kontext
arbeiten möchte. Um die Tastatureingaben abzufangen muss sich der
Meterpreter in einem Prozess befinden, der unter dem Nutzer läuft,
dessen Eingaben abgefangen werden sollen.

\paragraph{Client}

Der Meterpreter-Client ist in Ruby geschrieben und läuft auf dem System
des Angreifers. Der Benutzer kann diesen über eine Shell steuern, der
Client sendet dann die entsprechenden Befehle an den Server. Mit Ruby
ist der komplette Client skriptbar und es steht auch eine IRb zur
Verfügung.

\subsubsection{Netzschnittelle}

Meterpreter benutzt ein komplexes System zur Kommunikation zwischen dem
Client und dem Server. Es steht normalerweise nur ein Socket für die
Kommunikation zur Verfügung und es sollen nicht beliebig viele
Systemsockets geöffnet werden, damit Meterpreter nicht so einfach
entdeckt wird. Deswegen können Kanäle definiert werden die sich ähnlich
wie unterschiedliche Ports verhalten. Über diese Kanäle werden Pakete,
die nach dem \glos{tlv} Verfahren verpackt wurden, geschickt. Es gibt
einige vordefinierte Typen für die auch Parser mitgebracht werden,
zusätzlich können Erweiterungen auch eigene Typen definieren. Die
standardmäßig mitgelieferten Parser können die Nullterminierung bei
Strings überprüfen und Integer von der Network-\glos{byteorder} in die
Host-\glos{byteorder} umwandeln. Diese Pakete werden normalerweise
verschachtelt, sodass jede Schicht die Pakete wieder nach dem
\glos{tlv}-Verfahren einpackt. Meterpreter Erweiterungen können diese
Schnittstelle nutzen, Meterpreter Skript müssen sich um diese
Schnittstelle nicht kümmern, da diese eine Schicht weiter oben laufen.

\subsection{Ablauf eines Angriffs}

Ein Angriff auf ein System, bei dem Meterpreter als Exploit verwendet
wird, läuft in mehreren Schritten ab. Wenn der Exploit erfolgreich
ausgeführt wurde, führt er den mit eingepackten Payload aus. Dieser
Payload ist meistens ein Stager, der relativ klein ist und nur den
richtigen Payload wie z.B. Meterpreter nachladen soll. Der Stager baut
dann eine Verbindung zu Metasploit auf und lädt die \textit{.dll} oder
\textit{.so} des Meterpreters über das Netz nach und legt sie im Speicher
des angegriffenen Prozesses ab und startet ihn. Der soeben gestartete
Meterpreter baut, wenn er sich \textit{reverse} verbinden soll, eine
Verbindung zum Angreifer über den Socket des Stagers auf. Nach dem
Verbindungsaufbau wird der Meterpreter-Server vom Client konfiguriert
und es wird automatisch die \texttt{stdapi}-Erweiterung nachgeladen.
Wenn Meterpreter Administratorenrechte auf dem angegriffenen System hat,
wird auch noch die \texttt{priv}-Erweiterung nachgeladen
\cite{offensive-security-meterpreter}.

\subsection{Meterpreter Skripte schreiben}

Meterpreter lässt sich mithilfe von Ruby Skripten. Metasploit liefert
schon viele nützliche Skripte mit (siehe~\ref{compositions:meterpreter:meterpreter-skripte})
die nur noch ausgeführt werden müssen, es ist aber auch relativ einfach
möglich eigene Skripte zu schreiben. Während der Entwicklung biete es
sich an den integrierten IRb zu benutzen. Dieser verhält sich wie der
normale Ruby-IRb, es gibt aber die Variable \texttt{client}, die das
Objekt dieser Meterpreter Session enthält. Mithilfe von Methoden auf
dieser Variable können nun Operationen auf dem angegriffenen Host
durchgeführt werden.

Folgender Codeschnippel gibt z.B. den Namen des Betriebssystems aus:
\begin{lstlisting}[language=bash]
  client.sys.config.sysinfo['OS']
  => "Windows XP (Build 2600, Service Pack 3)."
\end{lstlisting}

Mithilfe von Railgun lassen sich Funktionen auf der Windows-API aufrufen.
Um ein Fenster Anzuzeigen wird nur folgender Code benötigt:

\begin{lstlisting}[language=bash]
  client.railgun.user32.MessageBoxA(0,"Hello","world","MB_OK")
\end{lstlisting}

Wenn man komplexe Skripte erstellt hat, dann können diese im Verzeichnis
\path{/scripts/meterpreter/} abgelegt werden und via \texttt{run <Skriptname>
<Parameter>} gestartet werden.


\subsection{Nutzen für \f}

Mithilfe des Meterpreters können nach dem erfolgreichen Angriff eines
Host noch sehr viele Operationen ausgeführt werden. Meterpreter eignet
sich auch sehr gut dazu von einem schon übernommenen Host aus weitere
Ziel anzugreifen und ihn als Brückenkopf zu verwenden oder um
Informationen für weitere Angriffe auf solch einem Host zu sammeln.
Es lassen sich auch viele Aufgaben über Skripte automatisieren was uns
beim Ziel eines sich automatisch weiterverbreitenden Wurm weiterhelfen
kann. Nachteilig ist, dass der Meterpreter Server nicht autark läuft und
immer auf eine Verbindung zu einem Meterpreter-Client angewiesen ist, um
Operationen durchzuführen. Die Nutzung von Meterpreter setzt auch die
Nutzung von Metasploit voraus, dieses stellt aber auch kein großes
Problem dar.
