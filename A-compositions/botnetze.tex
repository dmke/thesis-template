\section{Botnetze}
\label{compositions:botnetze}

\subsection{Abstract}
Netzwerke, die aus gekaperten Rechnern unter der gemeinsamen Kontrolle eines 
Angreifers bestehen, gehören zu den größten Gefahren des Internets. Diese Netze 
werden für verteilte Denial-of-Service-Angriffe und zum Versenden von Spam 
verwendet.

Zuerst wird die Frage geklärt, was ein Bot ist und woher der Name stammt. In dem
Zusammenhang wird der Begriff Botnet erläutert. Anschließend folgt eine 
Übersicht aktiver Botnetze, sowie deren ungefähre Größe. Danach wird der 
Einsatzzweck von solchen Botnetzen beschrieben und wie mit Hilfe dieser 
Netzwerke viel Geld verdient werden kann. Der Hauptteil dieser Ausarbeitung 
beschäftigt sich mit den verwendeten Technologien in Botnetzen. Es werden unter 
anderem die Kommunikationsformen, die Verbreitungstrategien und die 
Selbstverteidigung der Bots näher betrachtet. Daraufhin wird aufgezeigt wie 
diese Botnetze erkannt und entfernt werden können. 

Am Ende des Referats wird der mögliche Nutzen bestimmter Bot-Technologien für 
FIDIUS aufgezeigt. 

\subsection{Definition Bot/Botnet}
Bei einem Bot handelt es sich um ein Computerprogramm, das auf einen
kompromittierten System installiert wird. Dieser Bot stellt dem Angreifer einen
Kommunikationskanal zur Verfügung um das System zu kontrollieren. Der Begriff Bot
ist eine verkürzte Form des englischen Wortes Robot. Ein Bot besitzt 
Eigenschaften eines Wurms um sich autonom zu verbreiten, sowie Eigenschaften 
eines Trojaners, um Befehle des Angreifers entgegenzunehmen. 

Werden mehrere Bots zu einem Netzwerk zusammengeschlossen, spricht man von einem
Botnetz.

\subsection{Übersicht aktiver Botnetze}
Die nachfolgende Tabelle~\ref{fig:botnetz_uebersicht} zeigt eine Übersicht der 
bisher bekannten aktiven  Botnetze bis zum Jahr 2009, sowie deren geschätzte 
Größe. Dabei kann ein Computer mit verschiedenen Bots infiziert sein \cite{de-wiki:botnetze}.

\begin{figure}
  \centering
  \includegraphics[width=\linewidth]{images/botnetz_uebersicht}
  \caption{Übersicht von aktiven Botnetzen}
  \label{fig:botnetz_uebersicht}
\end{figure} 

\subsection{Einsatzszenarien}
Um mit infizierten Computern Geld zu verdienen, gibt es mehrere Möglichkeiten.
DDoS-Attacken, Diebstahl vertraulicher Informationen, Spamversand, Phishing, 
betrügerische Generierung von Klicks sowie der Download von Adware
und Schadprogrammen. Mit einem Botnetz können alle oben genannten Angriffe 
gleichzeitig durchgeführt werden. Die Abbildung~\ref{fig:botnetz_einsatz}
veranschaulicht nochmal die möglichen Einsatzszenarien eines Botnetzes \cite{namestnikov}.

\begin{figure}
  \centering
  \includegraphics[width=\linewidth]{images/botnetz_einsatz}
  \caption{Geschäfte mit Botnetzen}
  \label{fig:botnetz_einsatz}
\end{figure} 

In den folgenden Abschnitten werden die drei populärsten Angriffarten beschrieben.

\subsubsection{DDoS-Attacken}
Eine Distributed Denial-of-Service-Attacke (DDoS-Attacke) ist ein Angriff auf ein
Computersystem mit dem Ziel, dieses zu überlasten, damit es nicht mehr in der
Lage ist, Anfragen zu bearbeiten. DDoS-Attacken sind eine effektive Waffe um
Konkurrenten auszuschalten.

\subsubsection{Diebstahl vertraulicher Informationen}
Vertraulichen Informationen wie Kreditkartennummern, Bankinformationen und 
Passwörter zu verschiedenen Diensten (E-Mail, FTP, Instant-Messenger), die auf 
Anwendercomputern gespeichert sind, können von dem Angreifer ausgespäht werden.
Die gestohlenen Informationen werden dann weiter verkauft oder von dem Angreifer
selbst missbraucht.

\subsubsection{Spamversand}
Die gekaperten Computer werden zum Versenden von unerwünschten Mitteilungen
(Spam-Mails) missbraucht. Der Versand von unerwünschten Mitteilungen ist eine 
der wichtigsten Funktionen von Botnetzen.


\subsection{Verwendete Technologien in Botnetzen}
In den folgenden Abschnitten werden Technologien beschrieben, die in Botnetzen
eingesetzt werden.
\subsubsection{Topologien}
Die Grafik~\ref{fig:botnetz_topologie} veranschaulicht die zwei 
möglichen Netzwerk-Topologien eines Botnetzes \cite{dahl}.
\begin{figure}
  \centering
  \includegraphics[width=\linewidth]{images/botnetz_topologie}
  \caption{Botnetz-Topologien}
  \label{fig:botnetz_topologie}
\end{figure}
Bei der zentralisierten Topologie verbinden sich alle
infizierten Computer mit einer Steuerungszentrale ($Command \& Control Centre$) oder
auch $C\&C$. Die Steuerungszentrale registriert neu hinzugekommene Bots in seiner
Datenbank, überwacht ihren Zustand und schickt ihnen Befehle.

Der Aufbau eines dezentralisierten Botnetzes ist in der Praxis aufwändig, da 
jeder neu infizierte Computer über eine Liste der Bots verfügen muss, mit denen 
er sich innerhalb des Botnetzes verbinden soll.

\subsubsection{Kommunikation}
Damit die Befehle des Botnetz-Betreibers alle infizierten Rechner erreichen, 
muss eine Verbindung zwischen den Bots und der Steuerungszentrale bestehen. 
Eine mögliche Kommunikationsart zwischen Bot und Steuerungszentrale ist der
\gls{acr:irc}-orientierte Datenaustausch. Dabei verbindet sich jeder 
infizierte Computer mit einem vordefinierten \gls{acr:irc}-Server. 
Danach empfängt der Bot über einen bestimmten \gls{acr:irc}-Kanal die Befehle des Botnetz-Betreibers.

Eine weitere Kommunikationsart zwischen Bot und Steuerungszentrale ist die 
\gls{acr:im}-orientierte. Diese Kommunikationsart ist nicht sehr populär, weil 
der Datentransfer über \gls{acr:im}-Dienste wie AOL, MSN oder ICQ abläuft
und jeder infizierte Computer dazu einen eigenen \gls{acr:im}-Account benötigt, weil
es nicht möglich ist, sich mit dem gleichen Account von unterschiedlichen 
Computern aus einzuloggen. Zudem verhindern die \gls{acr:im}-Anbieter die automatische
Generierung von Accounts.

Die Web-orientiert Kommunikationsart wird immer beliebter bei Botnetz-Betreibern,
weil solche Botnetze sich einfach einrichten lassen und auf viele Webserver
zurückgegriffen werden kann. Ebenso lässt sich das gesamte Botnetz komfortabel über
ein Web-Interface steuern. Der Bot verbindet sich mit dem Webserver, welcher
als Stererungszentrale dient, und nimmt Befehle entgegen \cite{kamluk}. 

\subsubsection{Scanning-Mechanismen}
Damit sich ein Bot möglichst schnell unentdeckt verbreiten kann, benötigt der
Bot eine Scanning-Strategie. Jeder Bot könnte zur Verbreitung genutzt werden.
Tatsächlich werden nur wenige Bots zum Scannen und Verbreiten genutzt, damit nur
einige Bots entdeckt werden und nicht das gesamte Botnetz. Eine mögliche
Strategie sich zu verbreiten ist das \emph{Hit"=list"=Scanning}. Der Bot 
sucht nicht selbst nach verwundbaren Systeme, sondern nutzt eine vordefinierte
Liste mit verwundbaren Zielsystemen. Diese Liste wird dann abgearbeitet und 
jedes enthaltene System wird angegriffen. Die Liste kann im Bot vorkompiliert
sein oder kann von einem Webserver nachgeladen werden.

Bei der \emph{Topological"=Scanning"=Strategie} verbreitet sich der Bot mittels
eines Peer"=to"=Peer"=Netzes. Dazu wird die Liste der verwundbaren Zielsystemem
von bekannten Peers benutzt.

Die \emph{Flash"=Scanning"=Strategie} bietet eine weitere Möglichkeit um
verwundbare Zielsysteme zu finden. Bei dieser Strategie besteht bereits
eine Liste mit verwundbaren Computersystemen oder sie wird dynamisch
erstellt in dem \glos{googledorks} dazu missbraucht werden. Diese Art der Verbreitung
ist am schnellsten, weil kein Scann mehr durchgeführt werden muss.

Nachdem ein System kompromittiert wurde, benötigen die Bots eine Reinfektions-Strategie, 
um bereits infizierte Systeme nicht nochmal zu scannen bzw. zu infizieren. Dies
erfordert allerdings eine Übersicht aller Bots im Bonetz. Diese 
Reinfektions-Strategie wird auch als \emph{Permutation Scanning} bezeichnet
\cite{baier}.

\subsubsection{Infektion-Mechanismen}
Bei den Infektions-Mechanismen wird zwischen interaktiver Infektion und
automatisierter Infektion unterschieden.

Bei interaktiven Infektionen findet eine direkte Interaktion mit dem
Benutzer des zu infizierenden Computers statt. Der Nutzer des angegriffenen Systems 
wird dazu verleitet infizierte Programme herunterzuladen und auszuführen. Wenn
der Benutzer dazu aufgefordert wird bestimmte Browser-Plugins zu installieren,
so bezeichnet man diese Art von Angriff \emph{Drive"=by"=download}. 

Bei automatisierten Infektionen werden Sicherheitslücken ausgenutzt. Man
unterscheidet zwischen Lokalen-Exploits und Remote-Exploits. Die Lokalen-Exploits
dienen in erster Linie dazu, die vorhanden Rechte auf dem infizierten System
auszuweiten. Die Remote-Exploits dienen zum Infizieren von Computersystemen
über das Internet oder Netzwerk. Bei den automatisierten Infektionen findet
keine Interaktion mit dem Benutzer statt. Der Benutzer merkt in der Regel nicht,
wenn der Computer infiziert wurde \cite{baier}.

\subsubsection{Update-Mechanismen}
Jeder neue Bot bringt einen Update-Mechanismus mit. Mit Hilfe dieser Update-Funktion 
kann der Botnetz-Betreiber Fehler in den Bots beseitigen oder den Bot um weitere 
Funktionalität erweitern. Ebenso wichtig wie die Fehlerbehebung und Erweiterung
des Bots ist die Aktualiserung der Signatur des Bots. Anhand von Signaturen
werden infizierte Programme von Anti-Viren Herstellern erkannt. Und wenn sich die 
Signatur ändert, so wird die Chance erhöht vor Anti-Viren Programmen unentdeckt
zu bleiben.

\subsubsection{Selbstverteitigung-Mechanismen}
Jeder Botnetz-Betreiber hat zwei Sicherheitsziele. Das erste Sicherheitsziel
ist die Stuerungszentrale ($Command \& Control Centre$). Dieser Server ist die 
zentrale Komponente im Botnetz. Diese muss besonders gegen Übernahme geschützt werden.
Eine Schutzmaßnahme vor Abschaltung oder Übernahme der Stuerungszentrale wäre
das verlagern des Serverstandorts ins Ausland wie z.B. Russland oder Panama.
Diese Länder bieten \emph{Bullet"=Proof"=Hosting} an. Das bedeutet, dass Anfragen
und Beschwerden direkt entsorgt werden und denen nicht nachgegangen wird.
Falls die IP-Adresse doch gesperrt werden sollte, so greift man alternativ auf DynDNS
zurück. Aktualisiert sich die IP-Adresse, so wird auch der DNS-Eintrag aktualisiert.
Die neueste Form dabei ist \emph{Fast-Flux}. Bei Fast-Flux werden einer Domain 
viele verschiedene IP-Adressen zugeordnet (Round-Robbin-DNS). Der Bot erhält zufällig 
eine Adresse. Allerdings führt diese Adresse nicht zur richtigen Stuerungszentrale,
sondern zu einem infiziertem Bot, der die Anfrage an den richtige Stuerungszentrale
weiterleitet. Mit dieser Technik bleibt die Stuerungszentrale unentdeckt, da
nur wenige Bots mit dieser kommunizieren.

Damit das Botnetz wachsen kann, müssen sich die Bots gegen Entfernung
vom System schützen. Das ist das zweite Sicherheitsziel eines Botnetz-Betreibers.
Die Bots müssen vor Anti-Viren Programmen geschützt werden. Dazu kann sich
der Bot als Root-Kit im System installieren und ist somit fast unsichtbar im System agieren.
Ebenso sollte der Bot eine Ausführung verhindern, wenn ein Honeypot, VMWare 
oder ein Debugger erkannt wurde. Falls Anti-Viren Programme auf dem infizierten
System entdeckt werden, so sollten diese bei Bedarf deaktiviert werden. Wenn sich
der Bot im System installiert hat, so sollte er sich unauffällig verhalten
(wenig CPU, Speicher und Netzverkehr verbrauchen), damit er weiter unentdeckt
bleibt.

\subsection{Erkennung von Botnetzen}
Botznetze können auf mehrere Arten erkannt werden. Eine Möglichkeit das Verhalten
von Bots zu untersuchen, bieten Honeynets. Honeynets bestehen aus mehreren Honeypots,
die mit miteinander vernetzt sind. Eine weitere Möglichkeit Botnetze zu entdecken
ist die Analyse des Netzwerkverkehrs. Es gibt verschiedene Verfahren die sich zur 
Analyse des Netzwerkverkehrs eignen. 

Die \emph{signaturbasierten Verfahren} untersuchen den Netzwerkverkehr auf eine 
Bot-Infektion. Snort setzt dieses Verfahren ein, um Angriffe zu erkennen.
Der Nachteil dieses Verfahren besteht darin, dass nur bekannte Bedrohungen erkannt
werden können. 

Die \emph{anomaliebasierten Verfahren} beobachten den Netzwerkverkehr auf
Unregelmäßigkeiten. Ein Beispiel hierfür wäre die Übertragung von großen
Mengen an Daten oder die Kommunikation auf unüblichen Ports.

Die \emph{DNS-basierten Verfahren} untersuchen die DNS-Anfragen zur 
Steuerungszentrale. Ein Botnetz kann durch hohen DNS-Verkehr aufgespürt werden.
Der Ansatz von Schonewille und Van Helmond beruht darauf, dass Bots Kontakt zu
deaktivierten $C \& C$-Server herstellen wollen. Dies führt zu einer großen Anzahl
wiederauftretender Namensfehler, welche auffällig werden.

\subsection{Möglicher Nutzen für FIDIUS}
Nach Analyse mehrerer Bots, stellte sich für mich herraus, dass Bots vor allem
auf Windows-Systemen verbreitet sind. Dies liegt nicht nur an der Architektur
von Windows, sondern auch daran, dass die Allgemeinheit fast ausschließlich
Windows nutzt. Für Bot-Entwickler lohnt sich der Aufwand nicht für andere
Betriebssysteme zu entwickeln. Außerdem müssen Bots klein und möglichst ohne
größere Abhängigkeiten auskommen, damit der Bot auf möglichst vielen Computern
ausgeführt werden kann. Dieses ist bei interpretierten Sprachen nicht gegeben,
weil nicht jeder Anwender diese installiert hat. Als Topologie kommt nur der
zentralisierte Ansatz in Frage. Alles andere ist nicht nicht kontrollierbar.
Möglich wäre es die Steuerungszentrale in Ruby zu schreiben, die Bots in C.
Die Bots bekämen keine eigene Intelligenz, sondern würden benötigte Teile von
der Steuerungszentrale nachladen.

