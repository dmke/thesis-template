\section{NMap Scripting Engine}
\label{sec:comp:nse}

Ausarbeitung des Referats vom Projektwochenende. Vortragende: Bernhard
Katzmarski

\subsection{Benutzung} Die Nmap-Scripting-Engine (NSE), die seit
Version 4.21 in Nmap integriert ist [1], sorgt dafür, dass sich die
Funktionalität von Nmap durch Scripts erweitern lässt. Zum Zeitpunkt
der Erstellung dieses Textes kommt Nmap mit ca. 80 vorinstallierten
Scripts und insgesamt 153 stehen auf der Entwicklerhomepage zum
Herunterladen bereit[2].

Nmap-Scripts sind in der Skriptsprache LUA geschrieben. Bei LUA
handelt es sich um eine objektorientierte, Perl-ähnliche Sprache.  Das
bemerkenswerte an LUA ist die Größe des Interpreters, der nur 120 KB
groß ist.

Gängige Nmap-Scripts werden über das Argument -sC gestartet oder
gezielt mit dem Argument --script ausgeführt, das sich auf mehrere
Arten nutzen lässt. Man kann Dateinamen, Verzeichnis oder
Script-Kategorie angeben und die Scripts wählen die ausgeführt werden
sollen. Manche Scripts erwarten Parameter. Diese lassen sich mit dem
Argument --script-args setzen. Mehrere Parameter werden durch Komma
getrennt.

\subsection{Kategorien}

Scripts werden in Kategorien eingeteilt.

\begin{tabelle}{.5,.5}{Hochverfügbarkeitsklassen}{tab:comp:nse}
       \hline
         auth & Ermitteln von Authentisierungs-Credentials (z.B. Bruteforce) \\
       \hline
         default & Standard-Skripte, die beim Aufruf von -sC ausgeführt werden \\
       \hline
         discovery & Auswertung zugänglicher Dienste (z.B. Titel eines HTML-Dokuments oder SNMP-Einträge) \\
       \hline
         external & Skripte, die zwecks Weiterverarbeitung Daten an externe Dienste schicken (z.B. whois) \\
       \hline
         intrusive & Intrusive Skripte, die das Zielsystem (negativ) beeinträchtigen könnten (z.B. hohe CPU-Auslastung) \\
        \hline
          malware & Überprüfung der Infektion von Malware (Viren und Würmer) \\
        \hline
          safe & Defensive Skripte, die keine intrusiven und destriktiven Zugriffe durchführen \\
        \hline
          version & Erweiterung zum Fingerprinting mit dem Schalter -sV \\
        \hline
          vuln & Identifikation spezifischer Verwundbarkeiten (ähnlich einem Vulnerability Scanner) \\
        \hline
\end{tabelle}
Quelle: http://www.scip.ch/?labs.20100507

\subsection{Aufbau von NSE-Scripts}

 NSE-Scripts können in derivative und nicht-derivative Scripts geteilt
werden. Derivative Scripts benutzen nur bereits bestehende Information
von Nmap und bereiten diese auf.  Nicht-derivative führen zusätzliche
Netzzugriffe durch, um bestehende Informationen zu verifizieren, oder
neue zu gewinnen.

\subsubsection{Head} Im Head werden allgemeine Informationen über das
Script abgelegt.

\begin{lstlisting}[language={}]
  description = [[ Dieses minimale Skript identifiziert Webdienste ]]
  author = "Marc Ruef"
  license = "(c) 2010 by scip AG"
  categories = {"default", "safe"}
\end{lstlisting}

\subsubsection{Rules} Die Rules des Scripts entscheiden darüber, ob
ein Script ausgeführt wird oder nicht. Selbst, wenn das Script
explizit über --script ausgewählt wurde, muss auch die
Ausführungsregel gelten. Es muss mindestens eine der folgenden Regeln
vom Script überschrieben werden:

\begin{lstlisting}[language={}]
  prerule() 
  hostrule(host)
  portrule(host, port)
  postrule()
\end{lstlisting}
  
Am gängisten sind die Portrules. Die meisten Scripts entscheiden
anhand des Ports ob sie ausgeführt werden.


\subsubsection{Action} Die Action ist der Einstiegspunkt in das
Script. Von dort aus lässt sich das Script beliebig programmieren.

\subsection{Beispiele}

\begin{lstlisting}[language={}]
nmap -p80 -script=http-enum localhost
  PORT   STATE SERVICE
  80/tcp open  http
  | http-enum:  
  |   /admin/: Admin directory
  |   /icons/: Icons and images
  |   /phpmyadmin/: phpMyAdmin
  |_  /tmp/: tmp
\end{lstlisting}

\begin{lstlisting}[language={}]
nmap -p80 -script=http-php-version localhost
  PORT   STATE SERVICE
  80/tcp open  http
  | http-php-version: Versions from logo query 
  | (less accurate): 5.3.1RC3, 5.3.1
  | Versions from credits query (more accurate): 5.3.1RC3, 5.3.1
  |_Version from header x-powered-by: PHP/5.3.1
\end{lstlisting}

\begin{lstlisting}[language={}]
nmap -p 80 --script=http-form-brute 
--script-args 'http-form-brute.path=/dvwa/login.php' 127.0.0.1

  PORT   STATE SERVICE REASON  VERSION
  80/tcp open  http    syn-ack Apache httpd 2.2.14 ((Unix) 
  DAV/2 mod_ssl/2.2.14 OpenSSL/0.9.8l PHP/5.3.1 mod_apreq2-20090110/2.7.1 
  mod_perl/2.0.4 Perl/v5.10.1)
  | http-form-brute:  
  |   Accounts
  |     admin:password => Login correct
  |   Statistics
  |_    Perfomed 2011 guesses in 9 seconds, average tps: 223
  Final times for host: srtt: 333 rttvar: 2912  to: 100000
\end{lstlisting}


\begin{lstlisting}[language={}]
nmap --script smb-check-vulns.nse -p 445 192.168.178.40
Host is up (0.00017s latency).
PORT    STATE SERVICE
445/tcp open  microsoft-ds
MAC Address: 00:16:41:A7:39:D6 (USI)

Host script results:
| smb-check-vulns:  
|   MS08-067: LIKELY VULNERABLE (host stopped responding)
|   Conficker: UNKNOWN; not Windows, or Windows with disabled browser service 
|   (CLEAN); or Windows with crashed browser service (possibly INFECTED).
| |  If you know the remote system is Windows, try rebooting it and scanning
| |_ again. (Error NT_STATUS_OBJECT_NAME_NOT_FOUND)
|   regsvc DoS: CHECK DISABLED (add '--script-args=unsafe=1' to run)
|_  SMBv2 DoS (CVE-2009-3103): CHECK DISABLED 
|    (add '--script-args=unsafe=1' to run)
\end{lstlisting}

\subsection{Fazit}

NMap ist ein bewährtes Tool, um Informationen über ein Netz zu
gewinnen. Die Funktionalität von NMap lässt sich mit Scripts
erweitern, allerdings gibt es bis auf die Scripts auf der offiziellen
Entwicklerhomepage kaum fertige Scripts. Es würde einiges an Aufwand
kosten neue Scripts zu erstellen oder die vorhandenen anzupassen.

Es existieren auch Scripts, die Bruteforce-Angriffe, oder andere
Penetrationtests durchführen, was die Tendenz erahnen lässt NMap zu
einem Vulnerability-Scanner auszubauen. Allerdings kann der derzeitige
Script-Umfang mit anderen Tools, wie beispielsweise OpenVas nicht
mithalten.

Scripts könnten allerdings gut verwendet werden, um die gewonnen
Information aus NMap automatisiert weiterverarbeiten zu können.

\subsection{Material}

\begin{itemize}
  \item NSE Tutorial, abgerufen am 12.10.2010,
      URL: \url{http://www.scip.ch/?labs.20100507}
  \item Offizielle Nmap Webseite, abgerufen am 12.10.2010,
      URL: \url{http://nmap.org/nsedoc/}
  \item Blogeintrag favorisierte NSE-Scripte, abgerufen am 12.10.2010,
      URL: \url{http://www.attackvector.org/favorite-nmap-nse-scripts/}
	\item Wikipedia Artikel zu \textit{LUA}, abgerufen am 12.10.2010,
	    URL: \url{http://de.wikipedia.org/wiki/Lua}
  \item NASL CVE-Support, abgerufen am 15.10.2010,
      URL: \url{http://www.virtualblueness.net/nasl.html\#tth_sEc5.2.3}
\end{itemize}
