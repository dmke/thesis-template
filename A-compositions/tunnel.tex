\section{Tunnel - Hole Punching}
\label{composition:tunnel} Im Rahmen der zweiten Hälfte des Projektes
\f ist das Referat \textbf{Tunnel - Hole-Punching} entstanden.
Das vorliegende Dokument ergänzt das Referat als entsprechende
Ausarbeitung.

Das Thema des Referates hatte seinen Ursprung in der Begründung der
Verwendung von Brückenköpfen, welche es gilt in der zweiten Hälfte des
Projektes einzusetzen. Brückenköpfe werden im Rahmen des Projektes,
als durch '\f' unter Kontrolle gebrachte Rechner verstanden, die
es ermöglichen, weitere Netze durch sich selber zu erschliessen und zu
benutzen. Im weitesten Sinne können Brückenköpfe als Gateways
verstanden werden.

Diese ermöglichte Kommunikation, auch über Brückenköpfe hinweg, findet
auf Rechnern statt, welche von '\f' erfolgreich kompromittiert
wurden. Um Kommunikationen zu Rechnern aufrecht erhalten zu können
müssen verbindungslose oder verbindungsorientierte Übertragungsdienste
bereitgestellt werden. Aus diesem Grund sah es das Referat, als auch
die Ausarbeitung vor sich auf VPN-Software zu konzentrieren. Da es im
Rahmen des Projektes jedoch schon Vorstellungen, Erfahrung und die
Nutzung von VPN-Software gab/gibt, wurde das Kernthema auf einen
anderen Schwerpunkt verlagert - dem des Hole Punching.

Hierbei handelt es sich um ein Verfahren welches beschreibt, wie
Verbindungen zwischen entfernten Netzen hergestellt werden können,
auch wenn die zu kommunizierenden Rechner jeweils hinter einer
restriktiven Firewall und/oder einem NAT-GW eingebunden sind - es
handelt sich um ein Spezialfall des Tunneling.

Im Folgenden wird kurz auf die Nachteile von VPN-Software in Bezug auf
die Verwendung innerhalb des Projektes \f eingegangen.  Es wird
ein demonstratives Beispiel aufgezeigt wie Firewalls und NATs durch
Hole Punching durchbrochen werden können.

Abschließend wird auf den möglichen Nutzen für FIDIUS näher
eingegangen.

\subsection{Problem} Um einen Datenaustausch zwischen zwei Rechnern
über das WAN zu ermöglichen, welche sich beide hinter einem NAT
befinden, benötigt es Maßnahmen, welche als nicht immer gegeben
angesehen werden können. Aus der Praxis kennt man gängige Verfahren
wie \texttt{portforwarding} oder \texttt{UPnP}, die es regelhaft
erlauben, gerichtete und bestimmte Kommunikation zuzulassen oder nicht
zu ermöglichen. Beim Einsatz dieser Techniken wird die korrekte
Konfiguration jedoch schon vorausgesetzt, um einen
Kommunikationsaustausch überhaupt zuzulassen.  Sind diese
Voraussetzungen erfüllt können gängige VPNs eingesetzt werden.
\subsection{Software als Lösung}
\label{sec:sw-solution} Gängige VPN-Software, so wie wir sie im
Projekt in Form von OpenVPN einsetzen, können Rechner sehr gut direkt
miteinander verbinden oder einen kontrollierten, integeren
Datenaustausch ermöglichen. Jedoch müsste diese Software auf den zu
kompromittierenden Rechnern installiert und konfiguriert werden. Die
Funktionalität ist, da es sich im Fall von OpenVPN um den
Quasi-Standard handelt, im Szenario '\f-Brückenkopf' von
ausreichend bis überdimensioniert zu bewerten.

Die hohe Funktionalität erhöht zudem die Fehlerwahrscheinlichkeit und
mindert nicht die Notwendigkeit der anzupassenden
Konfiguration. Hierzu gehören, neben der eventuellen Installation von
Tap- oder Tun-Devices, auch das Routing, so wie die schon erwähnte
Ausgangslage, dass Firewalls und NATs entsprechend angepasst sind.

Dies bezüglich gibt es auch Software wie \texttt{campagnol
\url{http://campagnol.sourceforge.net/}}, welche ohne entsprechende
Einstellung von Firewalls eine Kommunikation zwischen zwei Rechnern
ermöglicht.

\textit{Campagnol is a distributed IP-based VPN software able to open
new connections through NATs or firewalls without any configuration.}

Die Installation vorausgesetzt könnte diese spezielle Software für
eine Kommunikation zwischen Rechnern, jeweils hinter einem NAT,
erfolgreich eingesetzt werden, ohne dass Informationen oder
Konfigurationen bezüglich einer bestehenden Firewall oder eines NATs,
notwendig sind. Ausschlaggebend dabei ist die Technik und das bekannte
Prinzip, nicht die Software selber, so dass sich der Schwerpunkt des
Referates und dieser Ausarbeitung auf eben diese Technik \textbf{Hole
Punching} konzentriert, als weitere Software zu erläutern.

\subsection{Technik als Lösung}
\subsubsection{NAT - Network Address Translation} Spricht man vom
\textit{Hole Punching} und stützt sich auf dementsprechende Literatur,
beginnen sehr viele Quellen mit der Vorstellung der verschiedenen
NAT-Typen, so dass diese grundlegende Information über die bestehenden
Klassen der NATs an dieser Stelle nachgeholt werden soll.

Gemäß ihres Verhaltens, werden NATs durch vier Typen klassifiziert

\begin{enumerate}
\item Full Cone
\item Restricted Cone
\item Port Restricted Cone
\item Symmetric
\end{enumerate}

\paragraph{Full Cone NAT}

Bezogen auf Abbildung~\ref{fig:nat-full-cone} sendet ein privater Host
(PH) einen initialen Request zum Remote Host A.  Hierfür öffnet der
NAT-Router einen öffentlichen Endpunkt (öffentliche Adresse mit
Port). Jegliche Kommunikation oder Verbindung zu einem entfernten
Rechner vom Port des privaten Hosts wird auf den gleichen Port des
NAT-Routers \texttt{gemappt}. Bei einem Full Cone NAT kann nun jeder
Remote Host von jedem Quellport zum Private Host A kommunizieren, in
dem er den öffentlichen Endpunkt (öffentliche Adresse mit Port) des
NAT-Routers benutzt.

\begin{figure}
  \centering
  \includegraphics[width=9cm]{images/nat_full_cone.png}
  \caption{Full Cone NAT}
  \label{fig:nat-full-cone}  
  \small{Quelle: \url{http://sarwiki.informatik.hu-berlin.de/Image:NAT_full_cone.png}}
\end{figure}

\paragraph{Restricted Cone NAT}

Das Verhalten des Restricted Cone NATs~\ref{fig:nat-restricted-cone}
ist dem des Full Cone NATs~\ref{fig:nat-full-cone} nahezu
identisch. Es unterscheidet sich lediglich dadurch, dass nur eine
Kommunikation zu genau dem Remote Hosts erfolgen darf, zu dem der
initiale Request gesendet wurde.  Alle anderen, eventuellen Pakete von
Remote Hosts, würden je nach Konfirguration des NAT-Routers
\texttt{gedroped} oder \texttt{rejected}.

\begin{figure}
  \centering
  \includegraphics[width=9cm]{images/nat_restricted_cone.png}
  \caption{Restricted Cone NAT}
  \label{fig:nat-restricted-cone}
  \small{Quelle: \url{http://sarwiki.informatik.hu-berlin.de/Image:NAT_restricted_cone.png}}
\end{figure}

\paragraph{Port Restricted Cone NAT}

Diese Klassifizierung des NATs~\ref{fig:nat-port-restricted-cone}
agiert noch restriktivier, als die des Restricted Cone NATs, indem vom
Remote Host zusätzlich gefordert wird, dass jegliche Antwort vom
angegebenen Ziel-Port aus erfolgt.

\begin{figure}
  \centering
  \includegraphics[width=9cm]{images/nat_port_restricted_cone.png}
  \caption{Port Restricted Cone NAT}
  \label{fig:nat-port-restricted-cone}
  \small{Quelle: \url{http://sarwiki.informatik.hu-berlin.de/Image:NAT_port_restricted_cone.png}}
\end{figure}

\paragraph{Symmetric NAT}

Das Symmetric NAT~\ref{fig:nat-sym} verwendet die höchste
Restriktion. Hier wird zu jeder angeforderten verbindungsorientierten
oder verbindungslosen Kommunikation des Private Hosts A zu einem
Remote Host B, ein anderer gemappter Port als der Source-Port, auf dem
NAT-Router benutzt (vergleiche hierzu Full Cone NAT).

\begin{figure}
  \centering
  \includegraphics[width=9cm]{images/nat_symmetric.png}
  \caption{Symmetric NAT}
  \label{fig:nat-sym}
  \small{Quelle: \url{http://sarwiki.informatik.hu-berlin.de/Image:NAT_symmetric.png}}
\end{figure}


\subsubsection{Hole Punching}
Beim Hole Punching unterscheidet man zwischen UDP Hole Punching und
TCP Hole Punching. Da UDP auf einer verbindungslosen Kommunikation
beruht und dadurch das gängigere Hole Punching Verfahren beschreibt,
wird im Folgenden darauf eingegangen.

UDP Hole Punching erlaubt es zwischen zwei Rechnern eine direkte
peer-to-peer UDP Session aufzubauen. Unter Zuhilfenahme eines
Rendezvous Servers ist es darüber hinaus möglich eine Session
aufzubauen, obgleich sich beide Rechner hinter einem NAT befinden.
Diese Technik wurde laut RFC 3027 (Protocol Complications with the IP
Network Address Translator) bereits vor dem heute bekannten STUN
(siehe~\ref{sec:stun}) durch das proprietäre \texttt{Activision gaming
  protocol} angewandt.  Auch ohne Verwendung eines Rendezvous Servers
kann man das Prinzip und die Technik des Hole Punchings beispielhaft
aufzeigen.
\paragraph{Hole Punching ohne Rendezvous Server}
Im Folgenden sei ein Host A und Host B gegeben. Host B ist dabei
hinter einem Port Restricted NAT~\ref{fig:nat-port-restricted-cone}.
Wenn Host A eine Verbindung zu B aufbauen soll ist dies auf Grund des
NATs nicht möglich, da kein Port in das LAN zu Host B weitergeleitet
wird. Selbst wenn Host B einen Listener~\ref{lst:netcat} durch
\texttt{netcat} auf Port 69 UDP laufen läßt, ist dieser vom WAN nicht
erreichbar, da dass NAT dies nicht zulässt.

\begin{lstlisting}[language=bash]
nc -u -l -p 69
\end{lstlisting}
\label{lst:netcat}

Wenn nun Host B ein UDP-Datagramm von Port 69 zu Host A
(74.125.39.104) schickt
\begin{lstlisting}[language=bash]
hping3 -c 1 -2 -s 69 -p 54 74.125.39.104
\end{lstlisting}
\label{lst:hping}
wird dadurch genau Port 69 zu Host B in das LAN geöffnet, falls es
sich um ein \texttt{(Port) Restricted Cone NAT }handelt.  Versucht
darauf hin Host A nochmals ein UDP-Datagramm zu Host B zu schicken
gelingt dies, da Host B diesen Port selber in das eigene LAN geöffnet
hat.

\paragraph{Hole Punching mit Rendezvous Server}
\label{sec:hole-rendezvous}
Die weitaus praktikablere Variante des Einsatzes des UDP Hole
Punching, ist die unter der Verwendung eines Rendezvous Servers.  Der
Rendezvous Server agiert in dem beispielhaften Szenario als
Mittelsmann zwischen zwei Hosts A und B, welche jeweils hinter einem
NAT in einem LAN eingebunden sind. Abbildung~\ref{fig:rendezvous} soll
diesen Sachverhalt visualisieren.

\begin{figure}
  \centering
  \includegraphics[width=6cm]{images/rendezvous.png}
  \caption{Zwei Hosts, jeweils hinter NAT}
  \label{fig:rendezvous}
\end{figure}

Diese Konstellation von zwei verschiedenen Hosts/Clients und einem
Server ist ein Beispiel, welches in vielen Audio-Anwendungen in Form
von VOIP-Software wiederzufinden ist. Eines der bekannteren Programme
ist sicherlich Skype.  Skype, wie auch andere peer-to-peer-Anwendungen
bedienen sich genau dieser Technik und \textit{bohren} dadurch Löcher
in die eigene Firewall und übergehen ebenfalls multiple NATs.

Angenommen Client A möchte Client B anrufen. Beide sind schon jeweils
am Rendezvous Server (in diesem Fall der Skype Server) mit ihren Daten
angemeldet. Der Skype Server weiss somit wie die jeweils öffentlichen
IP-Adressen von Client A und B sind, sowie über welchen Port jeweils
die Audiodaten von Client A und B übertragen werden würden, da dies im
Anwendungsclient eingestellt ist.

Sobald Client A dem Skype Server seinen Wunsch mitgeteilt hat Client B
anzurufen, initiiert dieser einen Sessionaufbau von Client B zu
A. Client B öffnet dadurch einen Port in der eigenen Firewall
bzw. leitet alles was an diesen Port des NATs adressiert ist und von
der öffentlichen Adresse des Clients A kommt, an die private Adresse
des Clients B weiter.

Der Versuch des Sessionaufbaus von Client B zu Client A wird an der
Firewall oder an dem NAT des Clients A scheitern, da diese(r) an den
Port adressierte Pakete dropen oder rejecten wird. Doch ab diesen
Zeitpunkt kann Client A, durch die gewonnenen Informationen des Skype
Servers über Client B selber die Session beginnen, indem er die
öffentliche Adresse von B, sowie den von Client B selber geöffneten
Port als Ziel angibt. Das Gespräch kann somit zu stande kommen.

Diese Technik wird unter anderem auch als RFC-Standard unter dem Namen
STUN eingesetzt.

\subsection{STUN}
\label{sec:stun} Früher \texttt{Simple traversal of UDP through NATs}
durch RFC 3489 definiert, steht STUN heute für \texttt{Session
Traversal Utilities for NATs} (RFC 5389) und ist der Standard für
VOIP-Anwendungen wie sip-basierte Telefonie. Es bietet eine
Client-Server Struktur und basiert auf \texttt{requests} der Clients,
welche durch STUN-\texttt{responses} beantwortet werden. Viele grosse
Anbieter wie sipgate.de haben öffentliche STUN-Server welche das
unter~\ref{sec:hole-rendezvous} exemplarisch gezeigte Vorgehen
unterstützen und genau dafür ausgelegt sind. Da sich das eigentliche
Vorgehen bei der Verwendung von STUN nicht wesentlich von dem Beispiel
abhebt, soll an dieser Stelle nicht weiterführend darauf eingegangen
werden. Nichtsdestotrotz zeigt es die aktuelle Praktikabilität.

\subsection{Nutzen für FIDIUS} Unter Punkt~\ref{sec:sw-solution} wurde
bereits herausgestellt, dass die Verwendung von VPN-Software schwierig
zu realisieren sein wird, da Installations- und Konfigurationsprozesse
zu komplex für kompromittierte Rechner wirken, um Brückenköpfe
aufzubauen. Herausgestochen bei der Software ist \texttt{campagnol}
durch die Verwendung der Technik des Hole Punchings.

Hole Punching ist als Technik dann für \f interessant, falls NATs
oder Firewalls überwunden werden müssen, ohne Konfigurationsprozesse
durchführen zu müssen. In wie weit es konkret einsetzbar ist, wird
sich im Laufe der zweiten Halbzeit des Projektes zeigen.


\subsection{Quellen} 
\cite{rfc3027},
\cite{rfc3489},
\cite{rfc5389},
\cite{tunnel-p2pnat},
\cite{tunnel-skype},
\cite{tunnel-nat-traversal},
\cite{en-wiki:udp-hole},
\cite{en-wiki:nat-hole}
